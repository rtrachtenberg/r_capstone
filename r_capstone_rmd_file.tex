% Options for packages loaded elsewhere
\PassOptionsToPackage{unicode}{hyperref}
\PassOptionsToPackage{hyphens}{url}
%
\documentclass[
]{article}
\usepackage{amsmath,amssymb}
\usepackage{iftex}
\ifPDFTeX
  \usepackage[T1]{fontenc}
  \usepackage[utf8]{inputenc}
  \usepackage{textcomp} % provide euro and other symbols
\else % if luatex or xetex
  \usepackage{unicode-math} % this also loads fontspec
  \defaultfontfeatures{Scale=MatchLowercase}
  \defaultfontfeatures[\rmfamily]{Ligatures=TeX,Scale=1}
\fi
\usepackage{lmodern}
\ifPDFTeX\else
  % xetex/luatex font selection
\fi
% Use upquote if available, for straight quotes in verbatim environments
\IfFileExists{upquote.sty}{\usepackage{upquote}}{}
\IfFileExists{microtype.sty}{% use microtype if available
  \usepackage[]{microtype}
  \UseMicrotypeSet[protrusion]{basicmath} % disable protrusion for tt fonts
}{}
\makeatletter
\@ifundefined{KOMAClassName}{% if non-KOMA class
  \IfFileExists{parskip.sty}{%
    \usepackage{parskip}
  }{% else
    \setlength{\parindent}{0pt}
    \setlength{\parskip}{6pt plus 2pt minus 1pt}}
}{% if KOMA class
  \KOMAoptions{parskip=half}}
\makeatother
\usepackage{xcolor}
\usepackage[margin=1in]{geometry}
\usepackage{color}
\usepackage{fancyvrb}
\newcommand{\VerbBar}{|}
\newcommand{\VERB}{\Verb[commandchars=\\\{\}]}
\DefineVerbatimEnvironment{Highlighting}{Verbatim}{commandchars=\\\{\}}
% Add ',fontsize=\small' for more characters per line
\usepackage{framed}
\definecolor{shadecolor}{RGB}{248,248,248}
\newenvironment{Shaded}{\begin{snugshade}}{\end{snugshade}}
\newcommand{\AlertTok}[1]{\textcolor[rgb]{0.94,0.16,0.16}{#1}}
\newcommand{\AnnotationTok}[1]{\textcolor[rgb]{0.56,0.35,0.01}{\textbf{\textit{#1}}}}
\newcommand{\AttributeTok}[1]{\textcolor[rgb]{0.13,0.29,0.53}{#1}}
\newcommand{\BaseNTok}[1]{\textcolor[rgb]{0.00,0.00,0.81}{#1}}
\newcommand{\BuiltInTok}[1]{#1}
\newcommand{\CharTok}[1]{\textcolor[rgb]{0.31,0.60,0.02}{#1}}
\newcommand{\CommentTok}[1]{\textcolor[rgb]{0.56,0.35,0.01}{\textit{#1}}}
\newcommand{\CommentVarTok}[1]{\textcolor[rgb]{0.56,0.35,0.01}{\textbf{\textit{#1}}}}
\newcommand{\ConstantTok}[1]{\textcolor[rgb]{0.56,0.35,0.01}{#1}}
\newcommand{\ControlFlowTok}[1]{\textcolor[rgb]{0.13,0.29,0.53}{\textbf{#1}}}
\newcommand{\DataTypeTok}[1]{\textcolor[rgb]{0.13,0.29,0.53}{#1}}
\newcommand{\DecValTok}[1]{\textcolor[rgb]{0.00,0.00,0.81}{#1}}
\newcommand{\DocumentationTok}[1]{\textcolor[rgb]{0.56,0.35,0.01}{\textbf{\textit{#1}}}}
\newcommand{\ErrorTok}[1]{\textcolor[rgb]{0.64,0.00,0.00}{\textbf{#1}}}
\newcommand{\ExtensionTok}[1]{#1}
\newcommand{\FloatTok}[1]{\textcolor[rgb]{0.00,0.00,0.81}{#1}}
\newcommand{\FunctionTok}[1]{\textcolor[rgb]{0.13,0.29,0.53}{\textbf{#1}}}
\newcommand{\ImportTok}[1]{#1}
\newcommand{\InformationTok}[1]{\textcolor[rgb]{0.56,0.35,0.01}{\textbf{\textit{#1}}}}
\newcommand{\KeywordTok}[1]{\textcolor[rgb]{0.13,0.29,0.53}{\textbf{#1}}}
\newcommand{\NormalTok}[1]{#1}
\newcommand{\OperatorTok}[1]{\textcolor[rgb]{0.81,0.36,0.00}{\textbf{#1}}}
\newcommand{\OtherTok}[1]{\textcolor[rgb]{0.56,0.35,0.01}{#1}}
\newcommand{\PreprocessorTok}[1]{\textcolor[rgb]{0.56,0.35,0.01}{\textit{#1}}}
\newcommand{\RegionMarkerTok}[1]{#1}
\newcommand{\SpecialCharTok}[1]{\textcolor[rgb]{0.81,0.36,0.00}{\textbf{#1}}}
\newcommand{\SpecialStringTok}[1]{\textcolor[rgb]{0.31,0.60,0.02}{#1}}
\newcommand{\StringTok}[1]{\textcolor[rgb]{0.31,0.60,0.02}{#1}}
\newcommand{\VariableTok}[1]{\textcolor[rgb]{0.00,0.00,0.00}{#1}}
\newcommand{\VerbatimStringTok}[1]{\textcolor[rgb]{0.31,0.60,0.02}{#1}}
\newcommand{\WarningTok}[1]{\textcolor[rgb]{0.56,0.35,0.01}{\textbf{\textit{#1}}}}
\usepackage{longtable,booktabs,array}
\usepackage{calc} % for calculating minipage widths
% Correct order of tables after \paragraph or \subparagraph
\usepackage{etoolbox}
\makeatletter
\patchcmd\longtable{\par}{\if@noskipsec\mbox{}\fi\par}{}{}
\makeatother
% Allow footnotes in longtable head/foot
\IfFileExists{footnotehyper.sty}{\usepackage{footnotehyper}}{\usepackage{footnote}}
\makesavenoteenv{longtable}
\usepackage{graphicx}
\makeatletter
\def\maxwidth{\ifdim\Gin@nat@width>\linewidth\linewidth\else\Gin@nat@width\fi}
\def\maxheight{\ifdim\Gin@nat@height>\textheight\textheight\else\Gin@nat@height\fi}
\makeatother
% Scale images if necessary, so that they will not overflow the page
% margins by default, and it is still possible to overwrite the defaults
% using explicit options in \includegraphics[width, height, ...]{}
\setkeys{Gin}{width=\maxwidth,height=\maxheight,keepaspectratio}
% Set default figure placement to htbp
\makeatletter
\def\fps@figure{htbp}
\makeatother
\setlength{\emergencystretch}{3em} % prevent overfull lines
\providecommand{\tightlist}{%
  \setlength{\itemsep}{0pt}\setlength{\parskip}{0pt}}
\setcounter{secnumdepth}{-\maxdimen} % remove section numbering
\ifLuaTeX
  \usepackage{selnolig}  % disable illegal ligatures
\fi
\IfFileExists{bookmark.sty}{\usepackage{bookmark}}{\usepackage{hyperref}}
\IfFileExists{xurl.sty}{\usepackage{xurl}}{} % add URL line breaks if available
\urlstyle{same}
\hypersetup{
  pdftitle={Capstone Project: Abalone Age Prediction},
  pdfauthor={Roxie Trachtenberg},
  hidelinks,
  pdfcreator={LaTeX via pandoc}}

\title{Capstone Project: Abalone Age Prediction}
\author{Roxie Trachtenberg}
\date{2023-12-04}

\begin{document}
\maketitle

\hypertarget{abalone-age-prediction}{%
\section{Abalone Age Prediction}\label{abalone-age-prediction}}

\hypertarget{introduction}{%
\subsection{Introduction}\label{introduction}}

The abalone dataset, downloaded from the AppliedPredictiveModeling R
package (which can be accessed at this
\href{https://github.com/cran/AppliedPredictiveModeling/tree/master/data}{link}),
contains 4,177 instances of physical and qualitative observations of
this type of marine snail. The data originated from a 1995 study
conducted by UC Irvine and contains the following data:

\begin{longtable}[]{@{}
  >{\raggedright\arraybackslash}p{(\columnwidth - 6\tabcolsep) * \real{0.2466}}
  >{\raggedright\arraybackslash}p{(\columnwidth - 6\tabcolsep) * \real{0.2466}}
  >{\raggedright\arraybackslash}p{(\columnwidth - 6\tabcolsep) * \real{0.2466}}
  >{\raggedright\arraybackslash}p{(\columnwidth - 6\tabcolsep) * \real{0.2603}}@{}}
\toprule\noalign{}
\begin{minipage}[b]{\linewidth}\raggedright
Variable Name
\end{minipage} & \begin{minipage}[b]{\linewidth}\raggedright
Variable Type
\end{minipage} & \begin{minipage}[b]{\linewidth}\raggedright
Units
\end{minipage} & \begin{minipage}[b]{\linewidth}\raggedright
Description
\end{minipage} \\
\midrule\noalign{}
\endhead
\bottomrule\noalign{}
\endlastfoot
Sex & Categorical & NA & Split into 3 categories: M (Male), F (Female),
and I (Immature) \\
LongestShell & Continuous & millimeters & Abalone length measured from
ends of longest shell \\
Diameter & Continuous & millimeters & Abalone shell diameter
E(perpendicular to longest shell length) \\
Height & Continuous & millimeters & Abalone height, including meat
within shell \\
WholeWeight & Continuous & grams & Whole abalone weight \\
ShuckedWeight & Continuous & grams & Weight of abalone flesh (not
including exoskeleton or shell) \\
VisceraWeight & Continuous & grams & Gut weight after bleeding \\
ShellWeight & Continuous & grams & Shell weight only (after being
dried) \\
Rings & Integer & NA & Number of abalone rings observed, used to derive
age in years \\
\end{longtable}

According to the researchers that piloted the 1995 study, the age of an
abalone is determined by cutting and staining the shell, followed by
analysis under microscope, which is a cumbersome task. All information
regarding the origin of this data can be found in the UC Irvine Machine
Learning Repository archives,
\href{http://archive.ics.uci.edu/dataset/1/abalone}{here}. The goal of
this project is to see if the measurements already a part of this
dataset can be used to predict the age instead of using the
aforementioned time-consuming and invasive procedures.

\hypertarget{load-libraries}{%
\subsection{Load Libraries}\label{load-libraries}}

First, we need to install packages and load libraries of those that are
needed to complete the project and run all of the project code.

\begin{Shaded}
\begin{Highlighting}[]
\CommentTok{\# Install packages}
\NormalTok{package\_names }\OtherTok{\textless{}{-}} \FunctionTok{c}\NormalTok{(}\StringTok{"AppliedPredictiveModeling"}\NormalTok{, }
                   \StringTok{"ggplot2"}\NormalTok{, }\StringTok{"dplyr"}\NormalTok{, }\StringTok{"tidyverse"}\NormalTok{, }\StringTok{"caret"}\NormalTok{,}
                   \StringTok{"randomForest"}\NormalTok{, }\StringTok{"corrplot"}\NormalTok{, }\StringTok{"glmnet"}\NormalTok{, }\StringTok{"rmarkdown"}\NormalTok{,}
                   \StringTok{"RColorBrewer"}\NormalTok{, }\StringTok{"xgboost"}\NormalTok{, }\StringTok{"DAAG"}\NormalTok{, }\StringTok{"Metrics"}\NormalTok{)}
\CommentTok{\# Load libraries}
\NormalTok{libraries\_to\_load }\OtherTok{\textless{}{-}} \FunctionTok{c}\NormalTok{(}\StringTok{"AppliedPredictiveModeling"}\NormalTok{, }
                       \StringTok{"ggplot2"}\NormalTok{, }\StringTok{"dplyr"}\NormalTok{, }\StringTok{"tidyverse"}\NormalTok{, }\StringTok{"caret"}\NormalTok{,}
                       \StringTok{"randomForest"}\NormalTok{, }\StringTok{"corrplot"}\NormalTok{, }\StringTok{"glmnet"}\NormalTok{, }\StringTok{"rmarkdown"}\NormalTok{,}
                       \StringTok{"RColorBrewer"}\NormalTok{, }\StringTok{"xgboost"}\NormalTok{, }\StringTok{"DAAG"}\NormalTok{, }\StringTok{"Metrics"}\NormalTok{)}
\ControlFlowTok{for}\NormalTok{ (lib }\ControlFlowTok{in}\NormalTok{ libraries\_to\_load) \{}
  \FunctionTok{library}\NormalTok{(lib, }\AttributeTok{character.only =} \ConstantTok{TRUE}\NormalTok{)}
\NormalTok{\}}
\end{Highlighting}
\end{Shaded}

\begin{verbatim}
## 
## Attaching package: 'dplyr'
\end{verbatim}

\begin{verbatim}
## The following objects are masked from 'package:stats':
## 
##     filter, lag
\end{verbatim}

\begin{verbatim}
## The following objects are masked from 'package:base':
## 
##     intersect, setdiff, setequal, union
\end{verbatim}

\begin{verbatim}
## -- Attaching core tidyverse packages ------------------------ tidyverse 2.0.0 --
## v forcats   1.0.0     v stringr   1.5.0
## v lubridate 1.9.2     v tibble    3.2.1
## v purrr     1.0.1     v tidyr     1.3.0
## v readr     2.1.4     
## -- Conflicts ------------------------------------------ tidyverse_conflicts() --
## x dplyr::filter() masks stats::filter()
## x dplyr::lag()    masks stats::lag()
## i Use the ]8;;http://conflicted.r-lib.org/conflicted package]8;; to force all conflicts to become errors
## Loading required package: lattice
## 
## 
## Attaching package: 'caret'
## 
## 
## The following object is masked from 'package:purrr':
## 
##     lift
## 
## 
## randomForest 4.7-1.1
## 
## Type rfNews() to see new features/changes/bug fixes.
## 
## 
## Attaching package: 'randomForest'
## 
## 
## The following object is masked from 'package:dplyr':
## 
##     combine
## 
## 
## The following object is masked from 'package:ggplot2':
## 
##     margin
## 
## 
## corrplot 0.92 loaded
## 
## Loading required package: Matrix
## 
## 
## Attaching package: 'Matrix'
## 
## 
## The following objects are masked from 'package:tidyr':
## 
##     expand, pack, unpack
## 
## 
## Loaded glmnet 4.1-6
## 
## 
## Attaching package: 'xgboost'
## 
## 
## The following object is masked from 'package:dplyr':
## 
##     slice
## 
## 
## 
## Attaching package: 'Metrics'
## 
## 
## The following objects are masked from 'package:caret':
## 
##     precision, recall
\end{verbatim}

\begin{Shaded}
\begin{Highlighting}[]
\CommentTok{\# Load and examine abalone dataset}
\FunctionTok{data}\NormalTok{(abalone)}
\FunctionTok{dim}\NormalTok{(abalone)}
\end{Highlighting}
\end{Shaded}

\begin{verbatim}
## [1] 4177    9
\end{verbatim}

\begin{Shaded}
\begin{Highlighting}[]
\FunctionTok{str}\NormalTok{(abalone)}
\end{Highlighting}
\end{Shaded}

\begin{verbatim}
## 'data.frame':    4177 obs. of  9 variables:
##  $ Type         : Factor w/ 3 levels "F","I","M": 3 3 1 3 2 2 1 1 3 1 ...
##  $ LongestShell : num  0.455 0.35 0.53 0.44 0.33 0.425 0.53 0.545 0.475 0.55 ...
##  $ Diameter     : num  0.365 0.265 0.42 0.365 0.255 0.3 0.415 0.425 0.37 0.44 ...
##  $ Height       : num  0.095 0.09 0.135 0.125 0.08 0.095 0.15 0.125 0.125 0.15 ...
##  $ WholeWeight  : num  0.514 0.226 0.677 0.516 0.205 ...
##  $ ShuckedWeight: num  0.2245 0.0995 0.2565 0.2155 0.0895 ...
##  $ VisceraWeight: num  0.101 0.0485 0.1415 0.114 0.0395 ...
##  $ ShellWeight  : num  0.15 0.07 0.21 0.155 0.055 0.12 0.33 0.26 0.165 0.32 ...
##  $ Rings        : int  15 7 9 10 7 8 20 16 9 19 ...
\end{verbatim}

Upon initial observation, it looks like the dataset has 4,177
observations and 9 columns, with variable information that generally
matches what is in the documentation.

\hypertarget{data-cleaning}{%
\subsection{Data Cleaning}\label{data-cleaning}}

Now that we have our libraries and dataset loaded without error, we can
inspect the structure and contents of the data and identify any parts
that may require cleaning, formatting, and/or handling of outliers or
missing data.

First, we created a new column for ``Age'' (which is just the number of
rings + 1.5), removed the ``Rings'' column to avoid confusion and issues
with multicollinearity (Age and Rings would have a perfect correlation
of 1 if the ``Rings'' variable were left in), and renamed the ``Type''
variable to ``Sex'' for clarity.

\begin{Shaded}
\begin{Highlighting}[]
\NormalTok{abalone }\OtherTok{\textless{}{-}}\NormalTok{ abalone }\SpecialCharTok{\%\textgreater{}\%} \FunctionTok{mutate}\NormalTok{(}\AttributeTok{Age =}\NormalTok{ Rings }\SpecialCharTok{+} \FloatTok{1.5}\NormalTok{) }\SpecialCharTok{\%\textgreater{}\%} \FunctionTok{rename}\NormalTok{(}\AttributeTok{Sex =}\NormalTok{ Type) }\SpecialCharTok{\%\textgreater{}\%} \FunctionTok{select}\NormalTok{(}\SpecialCharTok{{-}}\NormalTok{Rings)}
\end{Highlighting}
\end{Shaded}

Next, we checked the minimum of each variable (ignoring the result of
the categorical variable, ``Sex''), to ensure that there were no ``0''
measurements.

\begin{Shaded}
\begin{Highlighting}[]
\FunctionTok{apply}\NormalTok{(abalone, }\DecValTok{2}\NormalTok{, min)}
\end{Highlighting}
\end{Shaded}

\begin{verbatim}
##           Sex  LongestShell      Diameter        Height   WholeWeight 
##           "F"       "0.075"       "0.055"       "0.000"      "0.0020" 
## ShuckedWeight VisceraWeight   ShellWeight           Age 
##      "0.0010"      "0.0005"      "0.0015"        " 2.5"
\end{verbatim}

It looks like the ``Height'' variable has at least one value of 0 that
should be removed. It does not make sense that an abalone would not have
a measurement for height and this could skew our results down the line
when we apply models to the data.

In addition, we know from a logical perspective that the shucked and
viscera weight cannot be larger than whole weight, so any observations
that show this are likely erroneous.

\begin{Shaded}
\begin{Highlighting}[]
\FunctionTok{sum}\NormalTok{(abalone}\SpecialCharTok{$}\NormalTok{ShuckedWeight }\SpecialCharTok{\textgreater{}=}\NormalTok{ abalone}\SpecialCharTok{$}\NormalTok{WholeWeight)}
\end{Highlighting}
\end{Shaded}

\begin{verbatim}
## [1] 4
\end{verbatim}

\begin{Shaded}
\begin{Highlighting}[]
\CommentTok{\# there are 4 values here that should be removed}

\FunctionTok{sum}\NormalTok{(abalone}\SpecialCharTok{$}\NormalTok{VisceraWeight }\SpecialCharTok{\textgreater{}=}\NormalTok{ abalone}\SpecialCharTok{$}\NormalTok{WholeWeight)}
\end{Highlighting}
\end{Shaded}

\begin{verbatim}
## [1] 0
\end{verbatim}

\begin{Shaded}
\begin{Highlighting}[]
\CommentTok{\# There are no values here that need to be removed}
\end{Highlighting}
\end{Shaded}

Now that we know which values need to be removed, let's generate a clean
dataset:

\begin{Shaded}
\begin{Highlighting}[]
\NormalTok{abalone\_clean }\OtherTok{\textless{}{-}}\NormalTok{ abalone }\SpecialCharTok{\%\textgreater{}\%} \FunctionTok{filter}\NormalTok{(ShuckedWeight }\SpecialCharTok{\textless{}=}\NormalTok{ WholeWeight) }\SpecialCharTok{\%\textgreater{}\%} \FunctionTok{filter}\NormalTok{(Height }\SpecialCharTok{!=} \DecValTok{0}\NormalTok{)}

\FunctionTok{dim}\NormalTok{(abalone\_clean)}
\end{Highlighting}
\end{Shaded}

\begin{verbatim}
## [1] 4171    9
\end{verbatim}

The clean dataset has 4,171 observations and 9 columns, meaning we
successfully removed the 6 erroneous observations.

\hypertarget{exploratory-data-analysis-eda}{%
\subsection{Exploratory Data Analysis
(EDA)}\label{exploratory-data-analysis-eda}}

\hypertarget{plot-formatting}{%
\subsubsection{Plot Formatting}\label{plot-formatting}}

Now that we have a clean dataset, let's get a feel for the data and how
it's distributed throughout and within features. Since we will be doing
a decent amount of plotting and graphing, let's first look into a
color-blind friendly palette to use throughout our analysis using the
RColorBrewer package.

\begin{Shaded}
\begin{Highlighting}[]
\FunctionTok{display.brewer.all}\NormalTok{(}\AttributeTok{colorblindFriendly =} \ConstantTok{TRUE}\NormalTok{)}
\end{Highlighting}
\end{Shaded}

\includegraphics{r_capstone_rmd_file_files/figure-latex/unnamed-chunk-6-1.pdf}

We will be using the ``Paired'' and ``Set2'' palettes, so let's note
their hexadecimal color names for future use using the following code:

\begin{Shaded}
\begin{Highlighting}[]
\FunctionTok{display.brewer.pal}\NormalTok{(}\AttributeTok{n =} \DecValTok{8}\NormalTok{, }\AttributeTok{name =} \StringTok{\textquotesingle{}Paired\textquotesingle{}}\NormalTok{)}
\end{Highlighting}
\end{Shaded}

\includegraphics{r_capstone_rmd_file_files/figure-latex/unnamed-chunk-7-1.pdf}

\begin{Shaded}
\begin{Highlighting}[]
\FunctionTok{brewer.pal}\NormalTok{(}\AttributeTok{n =} \DecValTok{8}\NormalTok{, }\AttributeTok{name =} \StringTok{"Paired"}\NormalTok{)}
\end{Highlighting}
\end{Shaded}

\begin{verbatim}
## [1] "#A6CEE3" "#1F78B4" "#B2DF8A" "#33A02C" "#FB9A99" "#E31A1C" "#FDBF6F"
## [8] "#FF7F00"
\end{verbatim}

\begin{Shaded}
\begin{Highlighting}[]
\FunctionTok{display.brewer.pal}\NormalTok{(}\AttributeTok{n =} \DecValTok{8}\NormalTok{, }\AttributeTok{name =} \StringTok{\textquotesingle{}Set2\textquotesingle{}}\NormalTok{)}
\end{Highlighting}
\end{Shaded}

\includegraphics{r_capstone_rmd_file_files/figure-latex/unnamed-chunk-7-2.pdf}

\begin{Shaded}
\begin{Highlighting}[]
\FunctionTok{brewer.pal}\NormalTok{(}\AttributeTok{n =} \DecValTok{8}\NormalTok{, }\AttributeTok{name =} \StringTok{"Set2"}\NormalTok{)}
\end{Highlighting}
\end{Shaded}

\begin{verbatim}
## [1] "#66C2A5" "#FC8D62" "#8DA0CB" "#E78AC3" "#A6D854" "#FFD92F" "#E5C494"
## [8] "#B3B3B3"
\end{verbatim}

Finally, let's define a shortcut for applying good general formatting to
plots to save processing time later.

\begin{Shaded}
\begin{Highlighting}[]
\NormalTok{gen\_formatting }\OtherTok{\textless{}{-}} \FunctionTok{theme}\NormalTok{(}\AttributeTok{plot.title =} \FunctionTok{element\_text}\NormalTok{(}\AttributeTok{face =} \StringTok{"bold"}\NormalTok{),}
                        \AttributeTok{axis.title.y =} \FunctionTok{element\_text}\NormalTok{(}\AttributeTok{margin =} \FunctionTok{unit}\NormalTok{(}\FunctionTok{c}\NormalTok{(}\DecValTok{0}\NormalTok{, }\DecValTok{20}\NormalTok{, }\DecValTok{0}\NormalTok{, }\DecValTok{0}\NormalTok{), }\StringTok{"pt"}\NormalTok{))}
\NormalTok{)}
\end{Highlighting}
\end{Shaded}

\hypertarget{univariate-eda}{%
\subsubsection{Univariate EDA}\label{univariate-eda}}

First, let's visualize the distribution of abalone weights in the full
dataset:

\begin{Shaded}
\begin{Highlighting}[]
\NormalTok{plot1 }\OtherTok{\textless{}{-}} \FunctionTok{ggplot}\NormalTok{(abalone\_clean, }\FunctionTok{aes}\NormalTok{(}\AttributeTok{x =}\NormalTok{ WholeWeight)) }\SpecialCharTok{+} 
  \FunctionTok{geom\_histogram}\NormalTok{(}\AttributeTok{fill =} \StringTok{"\#FDBF6F"}\NormalTok{, }\AttributeTok{color =} \StringTok{"\#FF7F00"}\NormalTok{, }\AttributeTok{alpha =} \FloatTok{0.7}\NormalTok{) }\SpecialCharTok{+}
  \FunctionTok{labs}\NormalTok{(}\AttributeTok{title =} \StringTok{"Distribution of Abalone Weight"}\NormalTok{,}
       \AttributeTok{x =} \StringTok{"Weight (grams)"}\NormalTok{,}
       \AttributeTok{y =} \StringTok{"Count"}\NormalTok{) }\SpecialCharTok{+}
\NormalTok{  gen\_formatting}

\NormalTok{plot1}
\end{Highlighting}
\end{Shaded}

\begin{verbatim}
## `stat_bin()` using `bins = 30`. Pick better value with `binwidth`.
\end{verbatim}

\includegraphics{r_capstone_rmd_file_files/figure-latex/unnamed-chunk-9-1.pdf}

It looks like whole weight of abalones in our dataset fall generally
below 3 grams, with a majority clustered between 0 and 1.5 grams. This
looks like a fairly normal distribution of weights throughout the
dataset.

Now, let's explore whether there are significant weight differences by
Sex:

\begin{Shaded}
\begin{Highlighting}[]
\CommentTok{\# Visualize distribution of abalone weights by Sex}
\NormalTok{plot2 }\OtherTok{\textless{}{-}} \FunctionTok{ggplot}\NormalTok{(abalone\_clean, }\FunctionTok{aes}\NormalTok{(}\AttributeTok{x =}\NormalTok{ Sex, }\AttributeTok{y =}\NormalTok{ WholeWeight, }\AttributeTok{fill =}\NormalTok{ Sex)) }\SpecialCharTok{+} 
  \FunctionTok{geom\_boxplot}\NormalTok{() }\SpecialCharTok{+} 
  \FunctionTok{labs}\NormalTok{(}\AttributeTok{title =} \StringTok{"Distribution of Abalone Weight by Sex"}\NormalTok{,}
       \AttributeTok{x =} \StringTok{"Sex"}\NormalTok{,}
       \AttributeTok{y =} \StringTok{"Weight (grams)"}\NormalTok{) }\SpecialCharTok{+}
  \FunctionTok{scale\_fill\_brewer}\NormalTok{(}\AttributeTok{palette =} \StringTok{"Set2"}\NormalTok{, }\AttributeTok{labels =} \FunctionTok{c}\NormalTok{(}\StringTok{"Female"}\NormalTok{, }\StringTok{"Immature"}\NormalTok{, }\StringTok{"Male"}\NormalTok{)) }\SpecialCharTok{+}
  \FunctionTok{scale\_x\_discrete}\NormalTok{(}\AttributeTok{labels =} \FunctionTok{c}\NormalTok{(}\StringTok{"Female"}\NormalTok{, }\StringTok{"Immature"}\NormalTok{, }\StringTok{"Male"}\NormalTok{)) }\SpecialCharTok{+}
\NormalTok{  gen\_formatting}
  
\NormalTok{plot2 }
\end{Highlighting}
\end{Shaded}

\includegraphics{r_capstone_rmd_file_files/figure-latex/unnamed-chunk-10-1.pdf}

Looking at this, it's pretty clear that there is a significant weight
difference from Immature to mature abalones, but the male/female
difference is not immediately clear. Let's use a t-test to see if there
is a significant difference in weight between these two groups:

\begin{Shaded}
\begin{Highlighting}[]
\CommentTok{\# Conduct independent t{-}test: Are females significantly heavier than males?}

\CommentTok{\# First, test that homogeneity of variance is achieved (prerequisite for t{-}test)}
\NormalTok{abalone\_test }\OtherTok{\textless{}{-}}\NormalTok{ abalone\_clean }\SpecialCharTok{\%\textgreater{}\%} \FunctionTok{filter}\NormalTok{(Sex }\SpecialCharTok{\%in\%} \FunctionTok{c}\NormalTok{(}\StringTok{"M"}\NormalTok{, }\StringTok{"F"}\NormalTok{))}
\NormalTok{res }\OtherTok{\textless{}{-}} \FunctionTok{var.test}\NormalTok{(WholeWeight }\SpecialCharTok{\textasciitilde{}}\NormalTok{ Sex, }\AttributeTok{data =}\NormalTok{ abalone\_test)}
\NormalTok{res}
\end{Highlighting}
\end{Shaded}

\begin{verbatim}
## 
##  F test to compare two variances
## 
## data:  WholeWeight by Sex
## F = 0.83619, num df = 1306, denom df = 1527, p-value = 0.0008249
## alternative hypothesis: true ratio of variances is not equal to 1
## 95 percent confidence interval:
##  0.7533735 0.9285163
## sample estimates:
## ratio of variances 
##          0.8361923
\end{verbatim}

The ratio of variances is around 0.84, showing that variance is similar
between male and female group and that we can move forward with our
t-test.

\begin{Shaded}
\begin{Highlighting}[]
\FunctionTok{t.test}\NormalTok{(WholeWeight }\SpecialCharTok{\textasciitilde{}}\NormalTok{ Sex, }\AttributeTok{var.equal =} \ConstantTok{TRUE}\NormalTok{, }\AttributeTok{data =}\NormalTok{ abalone\_test)}
\end{Highlighting}
\end{Shaded}

\begin{verbatim}
## 
##  Two Sample t-test
## 
## data:  WholeWeight by Sex
## t = 3.2305, df = 2833, p-value = 0.00125
## alternative hypothesis: true difference in means between group F and group M is not equal to 0
## 95 percent confidence interval:
##  0.02164586 0.08849956
## sample estimates:
## mean in group F mean in group M 
##       1.0465321       0.9914594
\end{verbatim}

The t-value is 3.2305. A higher absolute t-value indicates a larger
difference between the groups. This represents the number of standard
deviations that the sample mean (mean in group F minus mean in group M)
is from the null hypothesis mean (0, assuming no difference between the
groups). In summary, the low p-value (0.00125) and the 95\% confidence
interval (not including 0) suggest that there is evidence to reject the
null hypothesis and that there is a significant difference between
sample means in abalone female weight versus male weight.

We know there are differences in weight by Sex, but what about Age?
Let's now focus on our target variable, Age, and visualize weight by age
group:

\begin{Shaded}
\begin{Highlighting}[]
\CommentTok{\# Visualize average abalone weight by age group}
\NormalTok{plot3 }\OtherTok{\textless{}{-}}\NormalTok{ abalone\_clean }\SpecialCharTok{\%\textgreater{}\%} \FunctionTok{filter}\NormalTok{(Sex }\SpecialCharTok{\%in\%} \FunctionTok{c}\NormalTok{(}\StringTok{"M"}\NormalTok{, }\StringTok{"F"}\NormalTok{)) }\SpecialCharTok{\%\textgreater{}\%} \FunctionTok{group\_by}\NormalTok{(Age) }\SpecialCharTok{\%\textgreater{}\%} 
  \FunctionTok{summarise}\NormalTok{(}\AttributeTok{avg\_weight =} \FunctionTok{mean}\NormalTok{(WholeWeight)) }\SpecialCharTok{\%\textgreater{}\%} 
  \FunctionTok{ggplot}\NormalTok{(}\FunctionTok{aes}\NormalTok{(}\AttributeTok{x =}\NormalTok{ Age, }\AttributeTok{y =}\NormalTok{ avg\_weight)) }\SpecialCharTok{+} 
  \FunctionTok{geom\_point}\NormalTok{(}\AttributeTok{size =} \DecValTok{3}\NormalTok{, }\AttributeTok{shape =} \DecValTok{23}\NormalTok{, }\AttributeTok{color =} \StringTok{"\#FF7F00"}\NormalTok{, }\AttributeTok{fill =} \StringTok{"\#FDBF6F"}\NormalTok{) }\SpecialCharTok{+}
  \FunctionTok{labs}\NormalTok{(}\AttributeTok{title =} \StringTok{"Average Weight per Abalone Age Group"}\NormalTok{,}
       \AttributeTok{x =} \StringTok{"Age Group"}\NormalTok{,}
       \AttributeTok{y =} \StringTok{"Average Weight by Age Group (grams)"}\NormalTok{) }\SpecialCharTok{+}
\NormalTok{  gen\_formatting}

\NormalTok{plot3}
\end{Highlighting}
\end{Shaded}

\includegraphics{r_capstone_rmd_file_files/figure-latex/unnamed-chunk-13-1.pdf}

This plot shows a general increasing trend and direct relationship
between age group and average weight. However, there are some
funky-looking values in the top right-hand corner of the plot (for age
groups of around 18 yrs and above). Average is really only a good
central tendency measure when the data in each age group is normally
distributed and has a healthy sample size. It is possible that the
distribution of weight data in the higher age groups is not normally
distributed or has low sample size.

Let's investigate distribution of age in the data:

\begin{Shaded}
\begin{Highlighting}[]
\CommentTok{\# Visualize age distribution within the abalone dataset:}
\NormalTok{plot4 }\OtherTok{\textless{}{-}}\NormalTok{ abalone }\SpecialCharTok{\%\textgreater{}\%} \FunctionTok{select}\NormalTok{(}\SpecialCharTok{{-}}\NormalTok{Sex) }\SpecialCharTok{\%\textgreater{}\%} \FunctionTok{group\_by}\NormalTok{(Age) }\SpecialCharTok{\%\textgreater{}\%}  \FunctionTok{ggplot}\NormalTok{(}\FunctionTok{aes}\NormalTok{(Age)) }\SpecialCharTok{+} 
  \FunctionTok{geom\_bar}\NormalTok{(}\AttributeTok{fill =} \StringTok{"\#66C2A5"}\NormalTok{, }\AttributeTok{color =} \StringTok{"\#B3B3B3"}\NormalTok{) }\SpecialCharTok{+}
  \FunctionTok{labs}\NormalTok{(}\AttributeTok{title =} \StringTok{"Distribution of Abalone Age"}\NormalTok{,}
       \AttributeTok{x =} \StringTok{"Age"}\NormalTok{,}
       \AttributeTok{y =} \StringTok{"Count"}\NormalTok{) }\SpecialCharTok{+}
\NormalTok{  gen\_formatting}

\NormalTok{plot4}
\end{Highlighting}
\end{Shaded}

\includegraphics{r_capstone_rmd_file_files/figure-latex/unnamed-chunk-14-1.pdf}

Interesting - it looks like our initial hunch about low sample size was
correct and our dataset suffers from survivorship bias, in that there
are fewer abalones that live to older age and therefore, there is less
data to reliably predict abalones of older age. We will keep this
unbalanced dataset issue in mind as we run machine learning models.

Is there at least a somewhat even distribution of males and females in
each age group, even if sample size is lower in older age groups?

\begin{Shaded}
\begin{Highlighting}[]
\CommentTok{\# Visualize distribution of abalone age by Sex:}
\NormalTok{facet\_colors\_fill }\OtherTok{\textless{}{-}} \FunctionTok{c}\NormalTok{(}\StringTok{"\#A6CEE3"}\NormalTok{, }\StringTok{"\#FB9A99"}\NormalTok{, }\StringTok{"\#B2DF8A"}\NormalTok{)}
\NormalTok{facet\_colors\_color }\OtherTok{\textless{}{-}} \FunctionTok{c}\NormalTok{(}\StringTok{"\#1F78B4"}\NormalTok{, }\StringTok{"\#E31A1C"}\NormalTok{, }\StringTok{"\#33A02C"}\NormalTok{)}

\NormalTok{plot5 }\OtherTok{\textless{}{-}} \FunctionTok{ggplot}\NormalTok{(abalone\_clean, }\FunctionTok{aes}\NormalTok{(}\AttributeTok{x =}\NormalTok{ Age, }\AttributeTok{fill =} \FunctionTok{as.factor}\NormalTok{(Sex), }\AttributeTok{color =} \FunctionTok{as.factor}\NormalTok{(Sex))) }\SpecialCharTok{+}
  \FunctionTok{geom\_histogram}\NormalTok{(}\AttributeTok{binwidth =} \DecValTok{2}\NormalTok{, }\AttributeTok{alpha =} \FloatTok{0.7}\NormalTok{) }\SpecialCharTok{+}  \CommentTok{\# Custom bar color}
  \FunctionTok{scale\_fill\_manual}\NormalTok{(}\AttributeTok{values =} \FunctionTok{setNames}\NormalTok{(facet\_colors\_fill, }\FunctionTok{unique}\NormalTok{(abalone\_clean}\SpecialCharTok{$}\NormalTok{Sex)),}
                    \AttributeTok{name =} \StringTok{"Sex"}\NormalTok{,}
                    \AttributeTok{labels =} \FunctionTok{c}\NormalTok{(}\AttributeTok{F =} \StringTok{"Female"}\NormalTok{, }\AttributeTok{I =} \StringTok{"Immature"}\NormalTok{, }\AttributeTok{M =} \StringTok{"Male"}\NormalTok{)) }\SpecialCharTok{+}
  \FunctionTok{scale\_color\_manual}\NormalTok{(}\AttributeTok{values =} \FunctionTok{setNames}\NormalTok{(facet\_colors\_color, }\FunctionTok{unique}\NormalTok{(abalone\_clean}\SpecialCharTok{$}\NormalTok{Sex)),}
                     \AttributeTok{name =} \StringTok{"Sex"}\NormalTok{,}
                     \AttributeTok{labels =} \FunctionTok{c}\NormalTok{(}\AttributeTok{F =} \StringTok{"Female"}\NormalTok{, }\AttributeTok{I =} \StringTok{"Immature"}\NormalTok{, }\AttributeTok{M =} \StringTok{"Male"}\NormalTok{)) }\SpecialCharTok{+} 
  \FunctionTok{facet\_grid}\NormalTok{(Sex }\SpecialCharTok{\textasciitilde{}}\NormalTok{ ., }\AttributeTok{labeller =} \FunctionTok{as\_labeller}\NormalTok{(}\FunctionTok{c}\NormalTok{(}\AttributeTok{F =} \StringTok{"Female"}\NormalTok{, }\AttributeTok{I =} \StringTok{"Immature"}\NormalTok{, }\AttributeTok{M =} \StringTok{"Male"}\NormalTok{))) }\SpecialCharTok{+}
  \FunctionTok{labs}\NormalTok{(}\AttributeTok{title =} \StringTok{"Distribution of Abalone Age by Sex"}\NormalTok{,}
       \AttributeTok{x =} \StringTok{"Age"}\NormalTok{,}
       \AttributeTok{y =} \StringTok{"Count"}\NormalTok{) }\SpecialCharTok{+}
\NormalTok{  gen\_formatting}

\NormalTok{plot5}
\end{Highlighting}
\end{Shaded}

\includegraphics{r_capstone_rmd_file_files/figure-latex/unnamed-chunk-15-1.pdf}

It looks like there is a normal distribution of grown males and females
in each age group. Immature abalones show more of a right-skewed
distribution.

\hypertarget{multivariate-eda}{%
\subsubsection{Multivariate EDA}\label{multivariate-eda}}

Now that we have made observations about some key variables in our
dataset, such as, age, sex, and weight, let's explore their associations
with one another through a correlation plot:

\begin{Shaded}
\begin{Highlighting}[]
\NormalTok{cor\_table }\OtherTok{=}\NormalTok{ abalone\_clean }\SpecialCharTok{\%\textgreater{}\%} \FunctionTok{select}\NormalTok{(}\SpecialCharTok{{-}}\NormalTok{Sex) }\SpecialCharTok{\%\textgreater{}\%} \FunctionTok{cor}\NormalTok{()}
\NormalTok{corplot1 }\OtherTok{\textless{}{-}} \FunctionTok{corrplot}\NormalTok{(cor\_table, }\AttributeTok{method =} \StringTok{\textquotesingle{}number\textquotesingle{}}\NormalTok{, }\AttributeTok{diag =} \ConstantTok{FALSE}\NormalTok{) }\CommentTok{\# colorful number}
\end{Highlighting}
\end{Shaded}

\includegraphics{r_capstone_rmd_file_files/figure-latex/unnamed-chunk-16-1.pdf}

\begin{Shaded}
\begin{Highlighting}[]
\NormalTok{corplot1}
\end{Highlighting}
\end{Shaded}

\begin{verbatim}
## $corr
##               LongestShell  Diameter    Height WholeWeight ShuckedWeight
## LongestShell     1.0000000 0.9867715 0.8277850   0.9252388     0.8993070
## Diameter         0.9867715 1.0000000 0.8339847   0.9254302     0.8945553
## Height           0.8277850 0.8339847 1.0000000   0.8195659     0.7766709
## WholeWeight      0.9252388 0.9254302 0.8195659   1.0000000     0.9704625
## ShuckedWeight    0.8993070 0.8945553 0.7766709   0.9704625     1.0000000
## VisceraWeight    0.9028490 0.8995504 0.7985533   0.9663343     0.9329124
## ShellWeight      0.8983006 0.9059778 0.8192815   0.9558888     0.8840581
## Age              0.5557164 0.5736965 0.5568703   0.5394067     0.4211111
##               VisceraWeight ShellWeight       Age
## LongestShell      0.9028490   0.8983006 0.5557164
## Diameter          0.8995504   0.9059778 0.5736965
## Height            0.7985533   0.8192815 0.5568703
## WholeWeight       0.9663343   0.9558888 0.5394067
## ShuckedWeight     0.9329124   0.8840581 0.4211111
## VisceraWeight     1.0000000   0.9080402 0.5027982
## ShellWeight       0.9080402   1.0000000 0.6273681
## Age               0.5027982   0.6273681 1.0000000
## 
## $corrPos
##            xName         yName x y      corr
## 1   LongestShell      Diameter 1 7 0.9867715
## 2   LongestShell        Height 1 6 0.8277850
## 3   LongestShell   WholeWeight 1 5 0.9252388
## 4   LongestShell ShuckedWeight 1 4 0.8993070
## 5   LongestShell VisceraWeight 1 3 0.9028490
## 6   LongestShell   ShellWeight 1 2 0.8983006
## 7   LongestShell           Age 1 1 0.5557164
## 8       Diameter  LongestShell 2 8 0.9867715
## 9       Diameter        Height 2 6 0.8339847
## 10      Diameter   WholeWeight 2 5 0.9254302
## 11      Diameter ShuckedWeight 2 4 0.8945553
## 12      Diameter VisceraWeight 2 3 0.8995504
## 13      Diameter   ShellWeight 2 2 0.9059778
## 14      Diameter           Age 2 1 0.5736965
## 15        Height  LongestShell 3 8 0.8277850
## 16        Height      Diameter 3 7 0.8339847
## 17        Height   WholeWeight 3 5 0.8195659
## 18        Height ShuckedWeight 3 4 0.7766709
## 19        Height VisceraWeight 3 3 0.7985533
## 20        Height   ShellWeight 3 2 0.8192815
## 21        Height           Age 3 1 0.5568703
## 22   WholeWeight  LongestShell 4 8 0.9252388
## 23   WholeWeight      Diameter 4 7 0.9254302
## 24   WholeWeight        Height 4 6 0.8195659
## 25   WholeWeight ShuckedWeight 4 4 0.9704625
## 26   WholeWeight VisceraWeight 4 3 0.9663343
## 27   WholeWeight   ShellWeight 4 2 0.9558888
## 28   WholeWeight           Age 4 1 0.5394067
## 29 ShuckedWeight  LongestShell 5 8 0.8993070
## 30 ShuckedWeight      Diameter 5 7 0.8945553
## 31 ShuckedWeight        Height 5 6 0.7766709
## 32 ShuckedWeight   WholeWeight 5 5 0.9704625
## 33 ShuckedWeight VisceraWeight 5 3 0.9329124
## 34 ShuckedWeight   ShellWeight 5 2 0.8840581
## 35 ShuckedWeight           Age 5 1 0.4211111
## 36 VisceraWeight  LongestShell 6 8 0.9028490
## 37 VisceraWeight      Diameter 6 7 0.8995504
## 38 VisceraWeight        Height 6 6 0.7985533
## 39 VisceraWeight   WholeWeight 6 5 0.9663343
## 40 VisceraWeight ShuckedWeight 6 4 0.9329124
## 41 VisceraWeight   ShellWeight 6 2 0.9080402
## 42 VisceraWeight           Age 6 1 0.5027982
## 43   ShellWeight  LongestShell 7 8 0.8983006
## 44   ShellWeight      Diameter 7 7 0.9059778
## 45   ShellWeight        Height 7 6 0.8192815
## 46   ShellWeight   WholeWeight 7 5 0.9558888
## 47   ShellWeight ShuckedWeight 7 4 0.8840581
## 48   ShellWeight VisceraWeight 7 3 0.9080402
## 49   ShellWeight           Age 7 1 0.6273681
## 50           Age  LongestShell 8 8 0.5557164
## 51           Age      Diameter 8 7 0.5736965
## 52           Age        Height 8 6 0.5568703
## 53           Age   WholeWeight 8 5 0.5394067
## 54           Age ShuckedWeight 8 4 0.4211111
## 55           Age VisceraWeight 8 3 0.5027982
## 56           Age   ShellWeight 8 2 0.6273681
## 
## $arg
## $arg$type
## [1] "full"
\end{verbatim}

This plot indicates that we will need to proceed cautiously regarding
multicollinearity when running machine learning algorithms. However,
features that are highly correlated with the age (e.g., ShellWeight,
Diameter) might be good candidates for inclusion in our machine learning
models over others. Viscera weight and whole weight, among other
measurements like this, are highly correlated, so this may provide a
good case for using interaction terms or dimensionality reduction prior
to running machine learning models. Finally, it is important to note
that higher correlations do not necessarily imply causation or better
predictive power in a machine learning context.

\hypertarget{machine-learning-models}{%
\subsection{Machine Learning Models}\label{machine-learning-models}}

\hypertarget{linear-regression}{%
\subsubsection{Linear Regression}\label{linear-regression}}

Now that we have a good sense of our cleaned up data, let's start
running some models!

First, let's define our RMSE function, set the seed (so that anyone that
reviews this will get the same indices and therefore model results that
I do when running my code below), and split our data into training and
test sets:

\begin{Shaded}
\begin{Highlighting}[]
\CommentTok{\# Define RMSE function}
\NormalTok{RMSE }\OtherTok{\textless{}{-}} \ControlFlowTok{function}\NormalTok{(true\_ratings, predicted\_ratings) \{}
  \FunctionTok{sqrt}\NormalTok{(}\FunctionTok{mean}\NormalTok{((true\_ratings }\SpecialCharTok{{-}}\NormalTok{ predicted\_ratings)}\SpecialCharTok{\^{}}\DecValTok{2}\NormalTok{))}
\NormalTok{\}}

\FunctionTok{set.seed}\NormalTok{(}\DecValTok{100}\NormalTok{) }\CommentTok{\# Set the seed}

\NormalTok{train\_index }\OtherTok{\textless{}{-}} \FunctionTok{createDataPartition}\NormalTok{(abalone\_clean}\SpecialCharTok{$}\NormalTok{Age, }\AttributeTok{p =} \FloatTok{0.8}\NormalTok{, }\AttributeTok{list =} \ConstantTok{FALSE}\NormalTok{)}
\NormalTok{train\_set }\OtherTok{\textless{}{-}}\NormalTok{ abalone\_clean[train\_index, ]}
\NormalTok{test\_set }\OtherTok{\textless{}{-}}\NormalTok{ abalone\_clean[}\SpecialCharTok{{-}}\NormalTok{train\_index, ]}
\end{Highlighting}
\end{Shaded}

Let's run a simple linear regression model (to use primarily for
comparison) with just ShellWeight (our variable most highly correlated
with Age according to the correlation plot above) as a predictor:

\begin{Shaded}
\begin{Highlighting}[]
\CommentTok{\# Prepare the model}
\NormalTok{lm\_fit }\OtherTok{\textless{}{-}}\NormalTok{ train\_set }\SpecialCharTok{\%\textgreater{}\%} \FunctionTok{lm}\NormalTok{(Age }\SpecialCharTok{\textasciitilde{}}\NormalTok{ ShellWeight, }\AttributeTok{data =}\NormalTok{ .)}

\CommentTok{\# Make predictions using the fitted model}
\NormalTok{pred }\OtherTok{\textless{}{-}} \FunctionTok{predict}\NormalTok{(lm\_fit, test\_set)}

\CommentTok{\# Generate root mean squared error (RMSE) value }
\NormalTok{rmse\_1 }\OtherTok{\textless{}{-}} \FunctionTok{RMSE}\NormalTok{(test\_set}\SpecialCharTok{$}\NormalTok{Age, pred)}

\CommentTok{\# Save RMSE value in a table to track progress later}
\NormalTok{rmse\_summary }\OtherTok{\textless{}{-}} \FunctionTok{tibble}\NormalTok{(}\AttributeTok{Model =} \StringTok{"Linear Reg {-} Shell Weight"}\NormalTok{, }\AttributeTok{RMSE =} \FunctionTok{round}\NormalTok{(rmse\_1, }\DecValTok{3}\NormalTok{))}
\NormalTok{rmse\_summary}
\end{Highlighting}
\end{Shaded}

\begin{verbatim}
## # A tibble: 1 x 2
##   Model                      RMSE
##   <chr>                     <dbl>
## 1 Linear Reg - Shell Weight   2.4
\end{verbatim}

Let's visualize our actual vs predicted values:

\begin{Shaded}
\begin{Highlighting}[]
\CommentTok{\# Visualize predicted vs actual values}
\NormalTok{plot\_data }\OtherTok{\textless{}{-}} \FunctionTok{data.frame}\NormalTok{(}\AttributeTok{Predicted\_value =}\NormalTok{ pred,   }
                        \AttributeTok{Observed\_value =}\NormalTok{ test\_set}\SpecialCharTok{$}\NormalTok{Age) }

\FunctionTok{ggplot}\NormalTok{(plot\_data, }\FunctionTok{aes}\NormalTok{(}\AttributeTok{x =}\NormalTok{ Observed\_value, }\AttributeTok{y =}\NormalTok{ Predicted\_value)) }\SpecialCharTok{+} 
  \FunctionTok{geom\_point}\NormalTok{() }\SpecialCharTok{+} 
  \FunctionTok{geom\_abline}\NormalTok{(}\AttributeTok{intercept =} \DecValTok{0}\NormalTok{, }\AttributeTok{slope =} \DecValTok{1}\NormalTok{, }\AttributeTok{color =} \StringTok{"\#66C2A5"}\NormalTok{, }\AttributeTok{linetype =} \StringTok{"dashed"}\NormalTok{) }\SpecialCharTok{+}
  \FunctionTok{labs}\NormalTok{(}\AttributeTok{title =} \StringTok{"Predicted vs Actual"}\NormalTok{,}
       \AttributeTok{x =} \StringTok{"Actual Values"}\NormalTok{,}
       \AttributeTok{y =} \StringTok{"Predicted Values"}\NormalTok{) }\SpecialCharTok{+}
\NormalTok{  gen\_formatting}
\end{Highlighting}
\end{Shaded}

\includegraphics{r_capstone_rmd_file_files/figure-latex/unnamed-chunk-19-1.pdf}

As we can see, the model has a harder job predicting the younger (likely
immature abalones) and older age groups correctly. We will experiment
with using weights to fix this issue of unbalanced data in the random
forest section. For now, let's try linear regression using all of the
variables (aside from Age) as predictors to see if we can improve our
result.

\begin{Shaded}
\begin{Highlighting}[]
\NormalTok{lm\_fit }\OtherTok{\textless{}{-}}\NormalTok{ train\_set }\SpecialCharTok{\%\textgreater{}\%} \FunctionTok{lm}\NormalTok{(Age }\SpecialCharTok{\textasciitilde{}}\NormalTok{ ., }\AttributeTok{data =}\NormalTok{ .)}
\NormalTok{pred }\OtherTok{\textless{}{-}} \FunctionTok{predict}\NormalTok{(lm\_fit, test\_set)}
\NormalTok{rmse\_2 }\OtherTok{\textless{}{-}} \FunctionTok{RMSE}\NormalTok{(test\_set}\SpecialCharTok{$}\NormalTok{Age, pred)}

\NormalTok{rmse\_summary }\OtherTok{\textless{}{-}} \FunctionTok{bind\_rows}\NormalTok{(rmse\_summary,}
                          \FunctionTok{tibble}\NormalTok{(}\AttributeTok{Model =} \StringTok{"Linear Reg {-} All Predictor Variables"}\NormalTok{,}
                                 \AttributeTok{RMSE =} \FunctionTok{round}\NormalTok{(rmse\_2, }\DecValTok{3}\NormalTok{)))}
\NormalTok{rmse\_summary}
\end{Highlighting}
\end{Shaded}

\begin{verbatim}
## # A tibble: 2 x 2
##   Model                                 RMSE
##   <chr>                                <dbl>
## 1 Linear Reg - Shell Weight             2.4 
## 2 Linear Reg - All Predictor Variables  2.12
\end{verbatim}

This improves our RMSE quite a bit! Now let's see if we can refine the
linear regression model further by adding interaction terms. Since the
dataset has minimal features and we are relatively familiar with how
each variable is related, we can incorporate interaction terms into our
model by multiplying two or more predictor variables that may be
correlated to capture their combined effects in the model. Interaction
terms allow the model to account for situations where the relationship
between one predictor variable and the response is related to the value
of another predictor variable (e.g., ShellWeight and WholeWeight).

\begin{Shaded}
\begin{Highlighting}[]
\NormalTok{train\_set\_intxns }\OtherTok{\textless{}{-}}\NormalTok{ train\_set }\SpecialCharTok{\%\textgreater{}\%} \FunctionTok{mutate}\NormalTok{(}
  \AttributeTok{physical\_measurements =}\NormalTok{ Height }\SpecialCharTok{*}\NormalTok{ Diameter }\SpecialCharTok{*}\NormalTok{ WholeWeight, }
  \AttributeTok{internal\_features =}\NormalTok{ ShuckedWeight }\SpecialCharTok{*}\NormalTok{ VisceraWeight,}
  \AttributeTok{shell\_features =}\NormalTok{ ShellWeight }\SpecialCharTok{*}\NormalTok{ Diameter)}

\NormalTok{test\_set\_intxns }\OtherTok{\textless{}{-}}\NormalTok{ test\_set }\SpecialCharTok{\%\textgreater{}\%} \FunctionTok{mutate}\NormalTok{(}
  \AttributeTok{physical\_measurements =}\NormalTok{ Height }\SpecialCharTok{*}\NormalTok{ Diameter }\SpecialCharTok{*}\NormalTok{ WholeWeight,}
  \AttributeTok{internal\_features =}\NormalTok{ ShuckedWeight }\SpecialCharTok{*}\NormalTok{ VisceraWeight,}
  \AttributeTok{shell\_features =}\NormalTok{ ShellWeight }\SpecialCharTok{*}\NormalTok{ Diameter)}

\NormalTok{lm\_fit }\OtherTok{\textless{}{-}}\NormalTok{ train\_set\_intxns }\SpecialCharTok{\%\textgreater{}\%} \FunctionTok{lm}\NormalTok{(Age }\SpecialCharTok{\textasciitilde{}}\NormalTok{ ., }\AttributeTok{data =}\NormalTok{ .)}
\NormalTok{pred }\OtherTok{\textless{}{-}} \FunctionTok{predict}\NormalTok{(lm\_fit, test\_set\_intxns)}
\NormalTok{rmse\_3 }\OtherTok{\textless{}{-}} \FunctionTok{RMSE}\NormalTok{(test\_set\_intxns}\SpecialCharTok{$}\NormalTok{Age, pred)}

\NormalTok{rmse\_summary }\OtherTok{\textless{}{-}} \FunctionTok{bind\_rows}\NormalTok{(rmse\_summary, }\FunctionTok{tibble}\NormalTok{(}
  \AttributeTok{Model =} \StringTok{"Linear Reg {-} All Predictor Variables + Interaction Terms"}\NormalTok{,}
  \AttributeTok{RMSE =} \FunctionTok{round}\NormalTok{(rmse\_3, }\DecValTok{3}\NormalTok{)))}
\NormalTok{rmse\_summary}
\end{Highlighting}
\end{Shaded}

\begin{verbatim}
## # A tibble: 3 x 2
##   Model                                                     RMSE
##   <chr>                                                    <dbl>
## 1 Linear Reg - Shell Weight                                 2.4 
## 2 Linear Reg - All Predictor Variables                      2.12
## 3 Linear Reg - All Predictor Variables + Interaction Terms  2.06
\end{verbatim}

Nice! We got an even better RMSE, but this does not solve our
multicollinearity issue (actually, adding in interaction terms may have
exacerbated it since most of the variables are pretty correlated and
combining them doesn't help with that). Let's see if we can address this
and any overfitting that may be occuring with regularization techniques.

\hypertarget{regularization-application}{%
\subsubsection{Regularization
Application}\label{regularization-application}}

\hypertarget{data-preparation}{%
\paragraph{Data Preparation}\label{data-preparation}}

First, let's transform Sex to a numeric value so that all predictor
variables are numeric:

\begin{Shaded}
\begin{Highlighting}[]
\NormalTok{train\_set\_continuous }\OtherTok{\textless{}{-}}\NormalTok{ train\_set\_intxns }\SpecialCharTok{\%\textgreater{}\%} 
  \FunctionTok{mutate}\NormalTok{(}\AttributeTok{Sex\_Integer =} \FunctionTok{as.integer}\NormalTok{(train\_set}\SpecialCharTok{$}\NormalTok{Sex)) }\SpecialCharTok{\%\textgreater{}\%} \FunctionTok{select}\NormalTok{(}\SpecialCharTok{{-}}\FunctionTok{c}\NormalTok{(Sex))}

\NormalTok{test\_set\_continuous }\OtherTok{\textless{}{-}}\NormalTok{ test\_set\_intxns }\SpecialCharTok{\%\textgreater{}\%} 
  \FunctionTok{mutate}\NormalTok{(}\AttributeTok{Sex\_Integer =} \FunctionTok{as.integer}\NormalTok{(test\_set}\SpecialCharTok{$}\NormalTok{Sex)) }\SpecialCharTok{\%\textgreater{}\%} \FunctionTok{select}\NormalTok{(}\SpecialCharTok{{-}}\FunctionTok{c}\NormalTok{(Sex))}
\end{Highlighting}
\end{Shaded}

Since regularization methods are sensitive to the scale of the input
features, let's (a) ensure our features are on a similar scale and (b)
if not, use a feature scaling technique such as standardization.

* Note that though the \texttt{glmnet} package that we are going to use
to run regularized models standardizes data automatically, we will do
this anyway for practice and educational purposes.

\begin{Shaded}
\begin{Highlighting}[]
\CommentTok{\# Initialize the scaled data frames}
\NormalTok{train\_set\_scaled }\OtherTok{\textless{}{-}}\NormalTok{ train\_set\_continuous}
\NormalTok{test\_set\_scaled }\OtherTok{\textless{}{-}}\NormalTok{ test\_set\_continuous}

\CommentTok{\# Ensure features are on a similar scale:}
\NormalTok{scale\_names\_train }\OtherTok{\textless{}{-}} \FunctionTok{colnames}\NormalTok{(train\_set\_continuous)[}\SpecialCharTok{!}\FunctionTok{colnames}\NormalTok{(train\_set\_continuous) }\SpecialCharTok{\%in\%} \FunctionTok{c}\NormalTok{(}\StringTok{"Age"}\NormalTok{)]}
\NormalTok{scale\_names\_test }\OtherTok{\textless{}{-}} \FunctionTok{colnames}\NormalTok{(test\_set\_continuous)[}\SpecialCharTok{!}\FunctionTok{colnames}\NormalTok{(test\_set\_continuous) }\SpecialCharTok{\%in\%} \FunctionTok{c}\NormalTok{(}\StringTok{"Age"}\NormalTok{)]}

\CommentTok{\# Check the means and standard deviations of the features}
\NormalTok{feature\_summary }\OtherTok{\textless{}{-}} \FunctionTok{data.frame}\NormalTok{(}
  \AttributeTok{Feature =}\NormalTok{ scale\_names\_train,}
  \AttributeTok{Mean =} \FunctionTok{colMeans}\NormalTok{(train\_set\_continuous[, scale\_names\_train]),}
  \AttributeTok{SD =} \FunctionTok{apply}\NormalTok{(train\_set\_continuous[, scale\_names\_train], }\DecValTok{2}\NormalTok{, sd)}
\NormalTok{)}

\NormalTok{feature\_summary}
\end{Highlighting}
\end{Shaded}

\begin{verbatim}
##                                     Feature       Mean         SD
## LongestShell                   LongestShell 0.52521120 0.11993128
## Diameter                           Diameter 0.40895297 0.09905777
## Height                               Height 0.14001947 0.04235368
## WholeWeight                     WholeWeight 0.83412433 0.49016970
## ShuckedWeight                 ShuckedWeight 0.36150989 0.22171133
## VisceraWeight                 VisceraWeight 0.18188212 0.10961135
## ShellWeight                     ShellWeight 0.23957070 0.13892691
## physical_measurements physical_measurements 0.06342414 0.05840722
## internal_features         internal_features 0.08838225 0.09587406
## shell_features               shell_features 0.11041915 0.08036662
## Sex_Integer                     Sex_Integer 2.05452367 0.82551401
\end{verbatim}

It does not look like our features are on a similar scale, indicated by
the wide range of means and standard deviations for each feature, so we
can scale using the code below:

\begin{Shaded}
\begin{Highlighting}[]
\CommentTok{\# Implement feature scaling using the scale() function}
\NormalTok{train\_set\_scaled[, scale\_names\_train] }\OtherTok{\textless{}{-}} \FunctionTok{scale}\NormalTok{(train\_set\_continuous[, scale\_names\_train])}
\NormalTok{test\_set\_scaled[, scale\_names\_test] }\OtherTok{\textless{}{-}} \FunctionTok{scale}\NormalTok{(test\_set\_continuous[, scale\_names\_test])}

\CommentTok{\# Now, check the means and standard deviations after scaling}
\NormalTok{scaled\_feature\_summary }\OtherTok{\textless{}{-}} \FunctionTok{data.frame}\NormalTok{(}
  \AttributeTok{Feature =}\NormalTok{ scale\_names\_train,}
  \AttributeTok{Mean =} \FunctionTok{colMeans}\NormalTok{(train\_set\_scaled[, scale\_names\_train]),}
  \AttributeTok{SD =} \FunctionTok{apply}\NormalTok{(train\_set\_scaled[, scale\_names\_train], }\DecValTok{2}\NormalTok{, sd)}
\NormalTok{)}

\NormalTok{scaled\_feature\_summary}
\end{Highlighting}
\end{Shaded}

\begin{verbatim}
##                                     Feature          Mean SD
## LongestShell                   LongestShell -4.459850e-16  1
## Diameter                           Diameter  1.995691e-16  1
## Height                               Height  1.453705e-16  1
## WholeWeight                     WholeWeight  1.006698e-16  1
## ShuckedWeight                 ShuckedWeight  1.095698e-16  1
## VisceraWeight                 VisceraWeight  6.999802e-17  1
## ShellWeight                     ShellWeight -3.797839e-17  1
## physical_measurements physical_measurements  4.854212e-17  1
## internal_features         internal_features -3.827461e-17  1
## shell_features               shell_features -7.299766e-17  1
## Sex_Integer                     Sex_Integer -3.604566e-17  1
\end{verbatim}

The scaling process has centered the data (resulting in a mean close to
zero) and standardized its spread (resulting in a standard deviation
close to 1) for each feature. Scaling ensures that the regularization
strength (the hyperparameter controlling the extent of regularization)
applies uniformly across features, so now that this is done, we can
proceed with regularization.

\hypertarget{regularization---ridge-l2-regression}{%
\paragraph{Regularization - Ridge (L2)
Regression}\label{regularization---ridge-l2-regression}}

Ridge (L2) regression works best when most variables in the model are
useful, as it shrinks parameters but does not remove them. Note that the
parameter that makes this Ridge (L2) Regression is the alpha parameter
set at 0. Let's see how our ridge regression model performs:

\begin{Shaded}
\begin{Highlighting}[]
\CommentTok{\# Define predictor names so that Age is not included as a predictor during model fitting}

\NormalTok{predictor\_names }\OtherTok{\textless{}{-}} \FunctionTok{colnames}\NormalTok{(train\_set\_scaled)[}\SpecialCharTok{!}\FunctionTok{colnames}\NormalTok{(train\_set\_scaled) }\SpecialCharTok{\%in\%} \FunctionTok{c}\NormalTok{(}\StringTok{"Age"}\NormalTok{)]}

\CommentTok{\# Run regularized model and use to output predictions}

\NormalTok{alpha0.fit }\OtherTok{\textless{}{-}} \FunctionTok{cv.glmnet}\NormalTok{(}\AttributeTok{x=} \FunctionTok{as.matrix}\NormalTok{(train\_set\_scaled[,predictor\_names]), }\AttributeTok{y=}\NormalTok{ train\_set\_scaled}\SpecialCharTok{$}\NormalTok{Age, }\AttributeTok{family=} \StringTok{"gaussian"}\NormalTok{, }
          \AttributeTok{type.measure =} \StringTok{"mse"}\NormalTok{, }\AttributeTok{alpha =} \DecValTok{0}\NormalTok{, }\AttributeTok{nlambda =} \DecValTok{100}\NormalTok{)}

\NormalTok{alpha0.predicted }\OtherTok{\textless{}{-}} \FunctionTok{predict}\NormalTok{(alpha0.fit, }\AttributeTok{s =}\NormalTok{ alpha0.fit}\SpecialCharTok{$}\NormalTok{lambda.min, }\AttributeTok{newx =} \FunctionTok{as.matrix}\NormalTok{(test\_set\_scaled[,predictor\_names]))}

\NormalTok{rmse\_4 }\OtherTok{\textless{}{-}} \FunctionTok{RMSE}\NormalTok{(test\_set\_scaled}\SpecialCharTok{$}\NormalTok{Age, alpha0.predicted)}

\NormalTok{rmse\_summary }\OtherTok{\textless{}{-}} \FunctionTok{bind\_rows}\NormalTok{(rmse\_summary,}
                          \FunctionTok{tibble}\NormalTok{(}\AttributeTok{Model =} \StringTok{"Linear Reg w Ridge (L2) Regression"}\NormalTok{,}
                                 \AttributeTok{RMSE =} \FunctionTok{round}\NormalTok{(rmse\_4, }\DecValTok{3}\NormalTok{)))}
\NormalTok{rmse\_summary}
\end{Highlighting}
\end{Shaded}

\begin{verbatim}
## # A tibble: 4 x 2
##   Model                                                     RMSE
##   <chr>                                                    <dbl>
## 1 Linear Reg - Shell Weight                                 2.4 
## 2 Linear Reg - All Predictor Variables                      2.12
## 3 Linear Reg - All Predictor Variables + Interaction Terms  2.06
## 4 Linear Reg w Ridge (L2) Regression                        2.17
\end{verbatim}

It looks like this didn't have quite the effect on RMSE that we were
hoping for, but maybe that's because Lasso (L1) Regression is a better
fit.

\hypertarget{regularization---lasso-l1-regression}{%
\paragraph{Regularization - Lasso (L1)
Regression}\label{regularization---lasso-l1-regression}}

Lasso (L1) Regression works best when there are some variables in the
model that are useless or redundant (sounds like our multicollinearity
issue!), as it shrinks these parameters completely to zero and therefore
generates a simpler model. Note that the parameter that indicates Lasso
(L1) Regression is the alpha parameter set at 1. Let's see how our lasso
regression model performs:

\begin{Shaded}
\begin{Highlighting}[]
\CommentTok{\# Run regularized model and use to output predictions}

\NormalTok{alpha1.fit }\OtherTok{\textless{}{-}} \FunctionTok{cv.glmnet}\NormalTok{(}\AttributeTok{x=} \FunctionTok{as.matrix}\NormalTok{(train\_set\_scaled[,predictor\_names]), }
                        \AttributeTok{y=}\NormalTok{ train\_set\_scaled}\SpecialCharTok{$}\NormalTok{Age, }\AttributeTok{family=} \StringTok{"gaussian"}\NormalTok{, }
                        \AttributeTok{type.measure =} \StringTok{"mse"}\NormalTok{, }\AttributeTok{alpha =} \DecValTok{1}\NormalTok{, }\AttributeTok{nlambda =} \DecValTok{100}\NormalTok{)}

\NormalTok{alpha1.predicted }\OtherTok{\textless{}{-}} \FunctionTok{predict}\NormalTok{(alpha1.fit, }\AttributeTok{s =}\NormalTok{ alpha1.fit}\SpecialCharTok{$}\NormalTok{lambda.min, }\AttributeTok{newx =} \FunctionTok{as.matrix}\NormalTok{(test\_set\_scaled[ , predictor\_names]))}

\NormalTok{rmse\_5 }\OtherTok{\textless{}{-}} \FunctionTok{RMSE}\NormalTok{(test\_set\_scaled}\SpecialCharTok{$}\NormalTok{Age, alpha1.predicted)}

\NormalTok{rmse\_summary }\OtherTok{\textless{}{-}} \FunctionTok{bind\_rows}\NormalTok{(rmse\_summary,}
                          \FunctionTok{tibble}\NormalTok{(}\AttributeTok{Model =} \StringTok{"Linear Reg w Lasso (L1) Regression"}\NormalTok{,}
                                 \AttributeTok{RMSE =} \FunctionTok{round}\NormalTok{(rmse\_5, }\DecValTok{3}\NormalTok{)))}
\NormalTok{rmse\_summary}
\end{Highlighting}
\end{Shaded}

\begin{verbatim}
## # A tibble: 5 x 2
##   Model                                                     RMSE
##   <chr>                                                    <dbl>
## 1 Linear Reg - Shell Weight                                 2.4 
## 2 Linear Reg - All Predictor Variables                      2.12
## 3 Linear Reg - All Predictor Variables + Interaction Terms  2.06
## 4 Linear Reg w Ridge (L2) Regression                        2.17
## 5 Linear Reg w Lasso (L1) Regression                        2.06
\end{verbatim}

The regularized model here seems to be an improvement, although it still
doesn't quite beat out the RMSE from our linear regression model with
interaction terms.

Finally, let's try elastic net regression to see if this improves our
RMSE.

\hypertarget{regularization---elastic-net-regression}{%
\paragraph{Regularization - Elastic Net
Regression}\label{regularization---elastic-net-regression}}

Elastic net regression combines the the two regularization penalties
(alpha and lambda) and is also good at dealing with situations when
there are correlations between parameters. Note that the parameter that
indicates Lasso (L1) Regression is the alpha parameter set between 0 and
1. We will use an alpha of 0.5 to get an idea of whether this method
will be useful.

\begin{Shaded}
\begin{Highlighting}[]
\CommentTok{\# Run regularized model and use to output predictions}

\NormalTok{alpha0.}\FloatTok{5.}\NormalTok{fit }\OtherTok{\textless{}{-}} \FunctionTok{cv.glmnet}\NormalTok{(}\AttributeTok{x=} \FunctionTok{as.matrix}\NormalTok{(train\_set\_scaled[ , predictor\_names]), }\AttributeTok{y=}\NormalTok{ train\_set\_scaled}\SpecialCharTok{$}\NormalTok{Age, }\AttributeTok{family=} \StringTok{"gaussian"}\NormalTok{, }
                        \AttributeTok{type.measure =} \StringTok{"mse"}\NormalTok{, }\AttributeTok{alpha =} \FloatTok{0.5}\NormalTok{, }\AttributeTok{nlambda =} \DecValTok{100}\NormalTok{)}

\NormalTok{alpha0.}\FloatTok{5.}\NormalTok{predicted }\OtherTok{\textless{}{-}} \FunctionTok{predict}\NormalTok{(alpha0.}\FloatTok{5.}\NormalTok{fit, }\AttributeTok{s =}\NormalTok{ alpha0.}\FloatTok{5.}\NormalTok{fit}\SpecialCharTok{$}\NormalTok{lambda.min, }\AttributeTok{newx =} \FunctionTok{as.matrix}\NormalTok{(test\_set\_scaled[ , predictor\_names]))}

\NormalTok{rmse\_6 }\OtherTok{\textless{}{-}} \FunctionTok{RMSE}\NormalTok{(test\_set\_scaled}\SpecialCharTok{$}\NormalTok{Age, alpha0.}\FloatTok{5.}\NormalTok{predicted)}

\NormalTok{rmse\_summary }\OtherTok{\textless{}{-}} \FunctionTok{bind\_rows}\NormalTok{(rmse\_summary,}
                          \FunctionTok{tibble}\NormalTok{(}\AttributeTok{Model =} \StringTok{"Elastic Net Regression"}\NormalTok{,}
                                 \AttributeTok{RMSE =} \FunctionTok{round}\NormalTok{(rmse\_6, }\DecValTok{3}\NormalTok{)))}
\NormalTok{rmse\_summary}
\end{Highlighting}
\end{Shaded}

\begin{verbatim}
## # A tibble: 6 x 2
##   Model                                                     RMSE
##   <chr>                                                    <dbl>
## 1 Linear Reg - Shell Weight                                 2.4 
## 2 Linear Reg - All Predictor Variables                      2.12
## 3 Linear Reg - All Predictor Variables + Interaction Terms  2.06
## 4 Linear Reg w Ridge (L2) Regression                        2.17
## 5 Linear Reg w Lasso (L1) Regression                        2.06
## 6 Elastic Net Regression                                    2.06
\end{verbatim}

It seems like the optimal combination of alpha and lambda may have yet
to be found.

\hypertarget{regularization---optimize-penalty-combination}{%
\paragraph{Regularization - Optimize Penalty
Combination}\label{regularization---optimize-penalty-combination}}

In order to understand which combination of penalties is truly best, we
need to try many different values for alpha by optimizing the elastic
net regression. We can do so and then run a model using the optimal
alpha value using the following code:

\begin{Shaded}
\begin{Highlighting}[]
\NormalTok{fit\_values }\OtherTok{\textless{}{-}} \FunctionTok{list}\NormalTok{() }\CommentTok{\# initialize empty list of fit\_values}

\CommentTok{\# for loop to generate models using different values of alpha}
\ControlFlowTok{for}\NormalTok{ (i }\ControlFlowTok{in} \DecValTok{0}\SpecialCharTok{:}\DecValTok{10}\NormalTok{) \{}
\NormalTok{  fit\_name }\OtherTok{\textless{}{-}} \FunctionTok{paste0}\NormalTok{(}\StringTok{"alpha"}\NormalTok{, i}\SpecialCharTok{/}\DecValTok{10}\NormalTok{)}
  
\NormalTok{  fit\_values[[fit\_name]] }\OtherTok{\textless{}{-}}
    \FunctionTok{cv.glmnet}\NormalTok{(}\AttributeTok{x =} \FunctionTok{as.matrix}\NormalTok{(train\_set\_continuous[ , predictor\_names]), }\AttributeTok{y=}\NormalTok{ train\_set\_continuous}\SpecialCharTok{$}\NormalTok{Age, }\AttributeTok{family=} \StringTok{"gaussian"}\NormalTok{, }
              \AttributeTok{type.measure =} \StringTok{"mse"}\NormalTok{, }\AttributeTok{alpha =}\NormalTok{ i}\SpecialCharTok{/}\DecValTok{10}\NormalTok{, }\AttributeTok{nlambda =} \DecValTok{100}\NormalTok{)}
\NormalTok{\}}

\NormalTok{results }\OtherTok{\textless{}{-}} \FunctionTok{data.frame}\NormalTok{() }\CommentTok{\# initialize empty dataframe for results}

\CommentTok{\# for loop to run models and showcase results}
\ControlFlowTok{for}\NormalTok{ (i }\ControlFlowTok{in} \DecValTok{0}\SpecialCharTok{:}\DecValTok{10}\NormalTok{) \{}
\NormalTok{  fit\_name }\OtherTok{\textless{}{-}} \FunctionTok{paste0}\NormalTok{(}\StringTok{"alpha"}\NormalTok{, i}\SpecialCharTok{/}\DecValTok{10}\NormalTok{)}
  
\NormalTok{  predicted }\OtherTok{\textless{}{-}} 
    \FunctionTok{predict}\NormalTok{(fit\_values[[fit\_name]], }
            \AttributeTok{s =}\NormalTok{ fit\_values[[fit\_name]]}\SpecialCharTok{$}\NormalTok{lambda.min, }\AttributeTok{newx =} \FunctionTok{as.matrix}\NormalTok{(test\_set\_continuous[ , predictor\_names]))}
  
\NormalTok{  rmse }\OtherTok{\textless{}{-}} \FunctionTok{RMSE}\NormalTok{(test\_set\_continuous}\SpecialCharTok{$}\NormalTok{Age, predicted)}
  
\NormalTok{  temp }\OtherTok{\textless{}{-}} \FunctionTok{data.frame}\NormalTok{(}\AttributeTok{alpha =}\NormalTok{ i}\SpecialCharTok{/}\DecValTok{10}\NormalTok{, }\AttributeTok{mse =}\NormalTok{ rmse, }\AttributeTok{fit\_name =}\NormalTok{ fit\_name)}
\NormalTok{  results }\OtherTok{\textless{}{-}} \FunctionTok{rbind}\NormalTok{(results, temp)}
\NormalTok{\}}

\NormalTok{results}
\end{Highlighting}
\end{Shaded}

\begin{verbatim}
##    alpha      mse fit_name
## 1    0.0 2.171615   alpha0
## 2    0.1 2.061348 alpha0.1
## 3    0.2 2.061860 alpha0.2
## 4    0.3 2.062202 alpha0.3
## 5    0.4 2.062729 alpha0.4
## 6    0.5 2.062729 alpha0.5
## 7    0.6 2.062844 alpha0.6
## 8    0.7 2.062910 alpha0.7
## 9    0.8 2.062947 alpha0.8
## 10   0.9 2.062865 alpha0.9
## 11   1.0 2.062894   alpha1
\end{verbatim}

\begin{Shaded}
\begin{Highlighting}[]
\CommentTok{\# rename colnames of results dataframe for clarity}
\NormalTok{results }\OtherTok{\textless{}{-}}\NormalTok{ results }\SpecialCharTok{\%\textgreater{}\%} \FunctionTok{rename}\NormalTok{(}\AttributeTok{rmse =}\NormalTok{ mse, }\AttributeTok{alphas =}\NormalTok{ alpha)}

\CommentTok{\# Print the best result:}
\NormalTok{results}\SpecialCharTok{$}\NormalTok{alphas[}\FunctionTok{which.min}\NormalTok{(results}\SpecialCharTok{$}\NormalTok{rmse)] }\CommentTok{\# best alpha is 0.4}
\end{Highlighting}
\end{Shaded}

\begin{verbatim}
## [1] 0.1
\end{verbatim}

\begin{Shaded}
\begin{Highlighting}[]
\CommentTok{\# Run elastic net regression model with alpha = 0.4}

\NormalTok{alpha0.}\FloatTok{4.}\NormalTok{fit }\OtherTok{\textless{}{-}} \FunctionTok{cv.glmnet}\NormalTok{(}\AttributeTok{x=} \FunctionTok{as.matrix}\NormalTok{(train\_set\_scaled[ , predictor\_names]), }\AttributeTok{y=}\NormalTok{ train\_set\_scaled}\SpecialCharTok{$}\NormalTok{Age, }\AttributeTok{family=} \StringTok{"gaussian"}\NormalTok{, }
                        \AttributeTok{type.measure =} \StringTok{"mse"}\NormalTok{, }\AttributeTok{alpha =} \FloatTok{0.4}\NormalTok{, }\AttributeTok{nlambda =} \DecValTok{100}\NormalTok{)}

\NormalTok{alpha0.}\FloatTok{4.}\NormalTok{predicted }\OtherTok{\textless{}{-}} \FunctionTok{predict}\NormalTok{(alpha0.}\FloatTok{4.}\NormalTok{fit, }\AttributeTok{s =}\NormalTok{ alpha0.}\FloatTok{4.}\NormalTok{fit}\SpecialCharTok{$}\NormalTok{lambda.min, }\AttributeTok{newx =} \FunctionTok{as.matrix}\NormalTok{(test\_set\_scaled[ , predictor\_names]))}

\NormalTok{rmse\_7 }\OtherTok{\textless{}{-}} \FunctionTok{RMSE}\NormalTok{(test\_set\_scaled}\SpecialCharTok{$}\NormalTok{Age, alpha0.}\FloatTok{4.}\NormalTok{predicted)}

\NormalTok{rmse\_summary }\OtherTok{\textless{}{-}} \FunctionTok{bind\_rows}\NormalTok{(rmse\_summary,}
                          \FunctionTok{tibble}\NormalTok{(}\AttributeTok{Model =} \StringTok{"Elastic Net Regression {-} Best Alpha"}\NormalTok{,}
                                 \AttributeTok{RMSE =} \FunctionTok{round}\NormalTok{(rmse\_7, }\DecValTok{3}\NormalTok{)))}
\NormalTok{rmse\_summary}
\end{Highlighting}
\end{Shaded}

\begin{verbatim}
## # A tibble: 7 x 2
##   Model                                                     RMSE
##   <chr>                                                    <dbl>
## 1 Linear Reg - Shell Weight                                 2.4 
## 2 Linear Reg - All Predictor Variables                      2.12
## 3 Linear Reg - All Predictor Variables + Interaction Terms  2.06
## 4 Linear Reg w Ridge (L2) Regression                        2.17
## 5 Linear Reg w Lasso (L1) Regression                        2.06
## 6 Elastic Net Regression                                    2.06
## 7 Elastic Net Regression - Best Alpha                       2.06
\end{verbatim}

It seems like an alpha value of 0.4 is best here. Even though the RMSE
is the same as the RMSE we got when running the initial elastic net
regression with alpha = 0.5, we will still add it to the table to show
that we tested the optimized model.

Overall, the elastic net regression seemed to improve our RMSE notably,
although still not as much as the linear regression with interaction
terms. Let's see if an ensemble method, which tends to be generally more
advanced and robust to overfitting, can help improve our RMSE.

\hypertarget{random-forest-model}{%
\subsubsection{Random Forest Model}\label{random-forest-model}}

\href{https://www.ibm.com/topics/random-forest}{According to IBM}, a
random forest model is a machine learning algorithm that generates a
combination, or ``forest,'' of decision trees, which act as a means to
split the data to arrive at a final predicted result. While decision
trees consider all the possible feature splits, random forests only
select a subset of those features.

Let's start by running a random forest model on our data with default
parameters:

\begin{Shaded}
\begin{Highlighting}[]
\NormalTok{rf\_model }\OtherTok{\textless{}{-}} \FunctionTok{randomForest}\NormalTok{(Age }\SpecialCharTok{\textasciitilde{}}\NormalTok{ ., }\AttributeTok{data =}\NormalTok{ train\_set\_intxns, }\AttributeTok{ntree =} \DecValTok{500}\NormalTok{)}
\NormalTok{predictions }\OtherTok{\textless{}{-}} \FunctionTok{predict}\NormalTok{(rf\_model, test\_set\_intxns)}
\NormalTok{rmse\_8 }\OtherTok{\textless{}{-}} \FunctionTok{RMSE}\NormalTok{(predictions, test\_set\_intxns}\SpecialCharTok{$}\NormalTok{Age)}

\NormalTok{rmse\_summary }\OtherTok{\textless{}{-}} \FunctionTok{bind\_rows}\NormalTok{(rmse\_summary,}
                          \FunctionTok{tibble}\NormalTok{(}\AttributeTok{Model =} \StringTok{"Random Forest {-} Default Params"}\NormalTok{,}
                                 \AttributeTok{RMSE =} \FunctionTok{round}\NormalTok{(rmse\_8, }\DecValTok{3}\NormalTok{)))}
\NormalTok{rmse\_summary}
\end{Highlighting}
\end{Shaded}

\begin{verbatim}
## # A tibble: 8 x 2
##   Model                                                     RMSE
##   <chr>                                                    <dbl>
## 1 Linear Reg - Shell Weight                                 2.4 
## 2 Linear Reg - All Predictor Variables                      2.12
## 3 Linear Reg - All Predictor Variables + Interaction Terms  2.06
## 4 Linear Reg w Ridge (L2) Regression                        2.17
## 5 Linear Reg w Lasso (L1) Regression                        2.06
## 6 Elastic Net Regression                                    2.06
## 7 Elastic Net Regression - Best Alpha                       2.06
## 8 Random Forest - Default Params                            2.04
\end{verbatim}

This is our best result yet! Let's try running the same random forest
model, but with incorporation of class weights (by assigning a higher
weight to older age groups with a lower sample size) to address the
issue of unbalanced data:

\begin{Shaded}
\begin{Highlighting}[]
\CommentTok{\# Assign class weights (higher weight for age groups 15 years and older)}
\NormalTok{class\_weights }\OtherTok{\textless{}{-}} \FunctionTok{ifelse}\NormalTok{(train\_set\_intxns}\SpecialCharTok{$}\NormalTok{Age }\SpecialCharTok{\textgreater{}=} \DecValTok{15}\NormalTok{, }\AttributeTok{yes =} \DecValTok{2}\NormalTok{, }\AttributeTok{no =} \DecValTok{1}\NormalTok{)}

\CommentTok{\# Run the model with class weights}
\NormalTok{rf\_model\_weights }\OtherTok{\textless{}{-}} \FunctionTok{randomForest}\NormalTok{(}
  \AttributeTok{formula =}\NormalTok{ Age }\SpecialCharTok{\textasciitilde{}}\NormalTok{ .,}
  \AttributeTok{data =}\NormalTok{ train\_set\_intxns,}
  \AttributeTok{ntree =} \DecValTok{500}\NormalTok{,}
  \AttributeTok{classwt =}\NormalTok{ class\_weights}
\NormalTok{  )}

\NormalTok{predictions }\OtherTok{\textless{}{-}} \FunctionTok{predict}\NormalTok{(rf\_model\_weights, test\_set\_intxns)}
\NormalTok{rmse\_rf\_wt }\OtherTok{\textless{}{-}} \FunctionTok{RMSE}\NormalTok{(predictions, test\_set\_intxns}\SpecialCharTok{$}\NormalTok{Age)}
\NormalTok{rmse\_rf\_wt}
\end{Highlighting}
\end{Shaded}

\begin{verbatim}
## [1] 2.035915
\end{verbatim}

It doesn't seem like this improved our RMSE from the initial random
forest model, so we will leave it out of the table to avoid clutter.

Note that multiple values (from 2 through 10) were tried in place of the
\texttt{yes\ =\ 2} argument in the code above, with no improvement in
RMSE.

Let's try tuning the model instead:

\begin{Shaded}
\begin{Highlighting}[]
\CommentTok{\# Specify the training control settings for cross{-}validation (with a 10{-}fold cross validation)}
\NormalTok{ctrl }\OtherTok{\textless{}{-}} \FunctionTok{trainControl}\NormalTok{(}\AttributeTok{method =} \StringTok{"cv"}\NormalTok{, }\AttributeTok{number =} \DecValTok{10}\NormalTok{)}

\CommentTok{\# We could use the tuneGrid argument of the train() function in the caret package to do this,}
\CommentTok{\# but let\textquotesingle{}s try adjusting parameters using the randomForest() function instead }

\NormalTok{param\_grid }\OtherTok{\textless{}{-}} \FunctionTok{expand.grid}\NormalTok{(}
  \AttributeTok{mtry =} \FunctionTok{c}\NormalTok{(}\DecValTok{2}\NormalTok{, }\DecValTok{4}\NormalTok{, }\DecValTok{6}\NormalTok{),}
  \AttributeTok{nodesize =} \FunctionTok{c}\NormalTok{(}\DecValTok{1}\NormalTok{,}\DecValTok{3}\NormalTok{,}\DecValTok{5}\NormalTok{),}
  \AttributeTok{ntree =} \FunctionTok{c}\NormalTok{(}\DecValTok{100}\NormalTok{, }\DecValTok{300}\NormalTok{, }\DecValTok{500}\NormalTok{)}
\NormalTok{)}

\CommentTok{\# Create an empty list to store the models}
\NormalTok{models }\OtherTok{\textless{}{-}} \FunctionTok{list}\NormalTok{()}
\end{Highlighting}
\end{Shaded}

Let's continue with our model tuning process. Note that the following
code may take around 5 minutes to run, depending on the capabilities of
your computer.

\begin{Shaded}
\begin{Highlighting}[]
\CommentTok{\# Iterate over each combination of parameters }
\CommentTok{\# }\AlertTok{WARNING}\CommentTok{: Note that this code will take around 5 minutes to run.}
\ControlFlowTok{for}\NormalTok{ (i }\ControlFlowTok{in} \DecValTok{1}\SpecialCharTok{:}\FunctionTok{nrow}\NormalTok{(param\_grid)) \{}
\NormalTok{  params }\OtherTok{\textless{}{-}}\NormalTok{ param\_grid[i, ]}
  
  \CommentTok{\# Train the model with the current set of parameters}
\NormalTok{  model }\OtherTok{\textless{}{-}} \FunctionTok{randomForest}\NormalTok{(}
    \AttributeTok{x =}\NormalTok{ train\_set\_intxns[, }\FunctionTok{colnames}\NormalTok{(train\_set\_intxns) }\SpecialCharTok{!=} \StringTok{"Age"}\NormalTok{],}
    \AttributeTok{y =}\NormalTok{ train\_set\_intxns}\SpecialCharTok{$}\NormalTok{Age,}
    \AttributeTok{mtry =}\NormalTok{ params}\SpecialCharTok{$}\NormalTok{mtry,}
    \AttributeTok{nodesize =}\NormalTok{ params}\SpecialCharTok{$}\NormalTok{nodesize,}
    \AttributeTok{ntree =}\NormalTok{ params}\SpecialCharTok{$}\NormalTok{ntree}
\NormalTok{  )}
  
  \CommentTok{\# Store the trained model}
\NormalTok{  models[[}\FunctionTok{paste}\NormalTok{(params}\SpecialCharTok{$}\NormalTok{mtry, params}\SpecialCharTok{$}\NormalTok{nodesize, params}\SpecialCharTok{$}\NormalTok{ntree, }\AttributeTok{sep =} \StringTok{"\_"}\NormalTok{)]] }\OtherTok{\textless{}{-}}\NormalTok{ model}
\NormalTok{\}}

\FunctionTok{names}\NormalTok{(models)}
\end{Highlighting}
\end{Shaded}

\begin{verbatim}
##  [1] "2_1_100" "4_1_100" "6_1_100" "2_3_100" "4_3_100" "6_3_100" "2_5_100"
##  [8] "4_5_100" "6_5_100" "2_1_300" "4_1_300" "6_1_300" "2_3_300" "4_3_300"
## [15] "6_3_300" "2_5_300" "4_5_300" "6_5_300" "2_1_500" "4_1_500" "6_1_500"
## [22] "2_3_500" "4_3_500" "6_3_500" "2_5_500" "4_5_500" "6_5_500"
\end{verbatim}

\begin{Shaded}
\begin{Highlighting}[]
\CommentTok{\# Create an empty vector to store the RMSE values}
\NormalTok{rmse\_values }\OtherTok{\textless{}{-}} \FunctionTok{numeric}\NormalTok{(}\FunctionTok{length}\NormalTok{(models))}

\CommentTok{\# Evaluate each model on the test set}
\ControlFlowTok{for}\NormalTok{ (i }\ControlFlowTok{in} \FunctionTok{seq\_along}\NormalTok{(models)) \{}
\NormalTok{  model }\OtherTok{\textless{}{-}}\NormalTok{ models[[i]]}
  
  \CommentTok{\# Make predictions on the test set}
\NormalTok{  predictions }\OtherTok{\textless{}{-}} \FunctionTok{predict}\NormalTok{(model, }\AttributeTok{newdata =}\NormalTok{ test\_set\_intxns[, }\FunctionTok{colnames}\NormalTok{(test\_set\_intxns) }\SpecialCharTok{!=} \StringTok{"Age"}\NormalTok{])}
  
  \CommentTok{\# Calculate RMSE}
\NormalTok{  rmse\_values[i] }\OtherTok{\textless{}{-}} \FunctionTok{RMSE}\NormalTok{(predictions, test\_set\_intxns}\SpecialCharTok{$}\NormalTok{Age)}
\NormalTok{\}}
\end{Highlighting}
\end{Shaded}

Now, let's find the best values and run our model again:

\begin{Shaded}
\begin{Highlighting}[]
\CommentTok{\# Find the index of the model with the lowest RMSE}
\NormalTok{best\_model\_index }\OtherTok{\textless{}{-}} \FunctionTok{which.min}\NormalTok{(rmse\_values)}

\CommentTok{\# Access the best model}
\NormalTok{best\_model }\OtherTok{\textless{}{-}}\NormalTok{ models[[best\_model\_index]]}
\FunctionTok{names}\NormalTok{(models[best\_model\_index])}
\end{Highlighting}
\end{Shaded}

\begin{verbatim}
## [1] "2_5_100"
\end{verbatim}

\begin{Shaded}
\begin{Highlighting}[]
\CommentTok{\# looks like the best parameters are mtry = 2, nodesize = 5, and ntree = 500}

\CommentTok{\# Print the RMSE values}
\NormalTok{rmse\_9 }\OtherTok{\textless{}{-}}\NormalTok{ rmse\_values[best\_model\_index]}
\NormalTok{rmse\_9}
\end{Highlighting}
\end{Shaded}

\begin{verbatim}
## [1] 2.026884
\end{verbatim}

\begin{Shaded}
\begin{Highlighting}[]
\NormalTok{rmse\_summary }\OtherTok{\textless{}{-}} \FunctionTok{bind\_rows}\NormalTok{(rmse\_summary,}
                          \FunctionTok{tibble}\NormalTok{(}\AttributeTok{Model =} \StringTok{"Random Forest {-} Tuned Params"}\NormalTok{,}
                                 \AttributeTok{RMSE =}\NormalTok{ rmse\_9))}
\NormalTok{rmse\_summary}
\end{Highlighting}
\end{Shaded}

\begin{verbatim}
## # A tibble: 9 x 2
##   Model                                                     RMSE
##   <chr>                                                    <dbl>
## 1 Linear Reg - Shell Weight                                 2.4 
## 2 Linear Reg - All Predictor Variables                      2.12
## 3 Linear Reg - All Predictor Variables + Interaction Terms  2.06
## 4 Linear Reg w Ridge (L2) Regression                        2.17
## 5 Linear Reg w Lasso (L1) Regression                        2.06
## 6 Elastic Net Regression                                    2.06
## 7 Elastic Net Regression - Best Alpha                       2.06
## 8 Random Forest - Default Params                            2.04
## 9 Random Forest - Tuned Params                              2.03
\end{verbatim}

It looks like using tuned parameters provided a slight improvement to
our model. Let's try another ensemble method for comparison purposes.

\hypertarget{gradient-boosting}{%
\subsubsection{Gradient Boosting}\label{gradient-boosting}}

\href{https://docs.aws.amazon.com/sagemaker/latest/dg/xgboost-HowItWorks.html}{According
to Amazon Web Services documentation}, gradient boosting trees attempt
to predict a target variable by combining estimates of a set of simpler
models. Although gradient boosted models can be prone to overfitting,
they have been known to perform better than random forests in some cases
due to the fact that they combine results along the way and iterate upon
results throughout the selection process, as opposed to combining
results at the end of the process as a random forest algorithm would.

Let's attempt to use a gradient boosted model with the \texttt{xgboost}
package.

First, we need to do a bit of pre-processing to get the data into a
format that is expected by \texttt{xgboost}:

\begin{Shaded}
\begin{Highlighting}[]
\CommentTok{\# Separate out x and y variables}

\NormalTok{x\_train }\OtherTok{\textless{}{-}} \FunctionTok{subset}\NormalTok{(train\_set\_continuous, }\AttributeTok{select =} \SpecialCharTok{{-}}\FunctionTok{c}\NormalTok{(Age))}
\NormalTok{y\_train }\OtherTok{\textless{}{-}}\NormalTok{ train\_set\_continuous}\SpecialCharTok{$}\NormalTok{Age}
\NormalTok{x\_test }\OtherTok{\textless{}{-}} \FunctionTok{subset}\NormalTok{(test\_set\_continuous, }\AttributeTok{select =} \SpecialCharTok{{-}}\FunctionTok{c}\NormalTok{(Age))}
\NormalTok{y\_test }\OtherTok{\textless{}{-}}\NormalTok{ test\_set\_continuous}\SpecialCharTok{$}\NormalTok{Age}

\CommentTok{\# Convert data to the format expected by xgboost}
\NormalTok{dtrain }\OtherTok{\textless{}{-}} \FunctionTok{xgb.DMatrix}\NormalTok{(}\AttributeTok{data =} \FunctionTok{as.matrix}\NormalTok{(x\_train), }\AttributeTok{label =}\NormalTok{ y\_train)}
\NormalTok{dtest }\OtherTok{\textless{}{-}} \FunctionTok{xgb.DMatrix}\NormalTok{(}\AttributeTok{data =} \FunctionTok{as.matrix}\NormalTok{(x\_test), }\AttributeTok{label =}\NormalTok{ y\_test)}
\end{Highlighting}
\end{Shaded}

Now, let's set some standard hyperparameters and run the boosted model:

\begin{Shaded}
\begin{Highlighting}[]
\CommentTok{\# Set hyperparameters (we may need to tune these)}
\NormalTok{params }\OtherTok{\textless{}{-}} \FunctionTok{list}\NormalTok{(}
  \AttributeTok{objective =} \StringTok{"reg:squarederror"}\NormalTok{,  }\CommentTok{\# Regression task}
  \AttributeTok{eval\_metric =} \StringTok{"rmse"}\NormalTok{,            }\CommentTok{\# Evaluation metric (Root Mean Squared Error)}
  \AttributeTok{max\_depth =} \DecValTok{2}\NormalTok{,                 }\CommentTok{\# Maximum depth of a tree}
  \AttributeTok{eta =} \FloatTok{0.1}\NormalTok{,               }\CommentTok{\# Learning rate}
  \AttributeTok{nrounds =} \DecValTok{400}     \CommentTok{\# Number of boosting rounds}
\NormalTok{)}

\CommentTok{\# Train the xgboost model}
\NormalTok{xgb\_model }\OtherTok{\textless{}{-}} \FunctionTok{xgboost}\NormalTok{(}\AttributeTok{data =}\NormalTok{ dtrain, }\AttributeTok{params =}\NormalTok{ params, }\AttributeTok{nrounds =} \DecValTok{100}\NormalTok{)}
\end{Highlighting}
\end{Shaded}

\begin{verbatim}
## [09:11:13] WARNING: src/learner.cc:767: 
## Parameters: { "nrounds" } are not used.
## 
## [1]  train-rmse:10.350786 
## [2]  train-rmse:9.383992 
## [3]  train-rmse:8.519721 
## [4]  train-rmse:7.748605 
## [5]  train-rmse:7.059475 
## [6]  train-rmse:6.447664 
## [7]  train-rmse:5.905184 
## [8]  train-rmse:5.423504 
## [9]  train-rmse:4.998685 
## [10] train-rmse:4.622758 
## [11] train-rmse:4.293458 
## [12] train-rmse:4.006589 
## [13] train-rmse:3.754900 
## [14] train-rmse:3.537011 
## [15] train-rmse:3.349443 
## [16] train-rmse:3.188586 
## [17] train-rmse:3.049491 
## [18] train-rmse:2.932306 
## [19] train-rmse:2.831326 
## [20] train-rmse:2.747182 
## [21] train-rmse:2.674444 
## [22] train-rmse:2.615269 
## [23] train-rmse:2.565044 
## [24] train-rmse:2.520583 
## [25] train-rmse:2.486170 
## [26] train-rmse:2.456851 
## [27] train-rmse:2.430185 
## [28] train-rmse:2.404959 
## [29] train-rmse:2.387562 
## [30] train-rmse:2.372900 
## [31] train-rmse:2.353281 
## [32] train-rmse:2.336760 
## [33] train-rmse:2.325060 
## [34] train-rmse:2.313444 
## [35] train-rmse:2.299671 
## [36] train-rmse:2.293200 
## [37] train-rmse:2.286799 
## [38] train-rmse:2.277265 
## [39] train-rmse:2.272616 
## [40] train-rmse:2.265476 
## [41] train-rmse:2.255701 
## [42] train-rmse:2.252419 
## [43] train-rmse:2.245623 
## [44] train-rmse:2.242596 
## [45] train-rmse:2.238161 
## [46] train-rmse:2.231346 
## [47] train-rmse:2.227226 
## [48] train-rmse:2.224825 
## [49] train-rmse:2.217675 
## [50] train-rmse:2.212466 
## [51] train-rmse:2.210721 
## [52] train-rmse:2.205878 
## [53] train-rmse:2.201025 
## [54] train-rmse:2.199086 
## [55] train-rmse:2.195317 
## [56] train-rmse:2.190158 
## [57] train-rmse:2.185106 
## [58] train-rmse:2.182368 
## [59] train-rmse:2.181071 
## [60] train-rmse:2.177423 
## [61] train-rmse:2.176131 
## [62] train-rmse:2.172269 
## [63] train-rmse:2.168171 
## [64] train-rmse:2.162905 
## [65] train-rmse:2.161383 
## [66] train-rmse:2.159236 
## [67] train-rmse:2.156410 
## [68] train-rmse:2.152131 
## [69] train-rmse:2.149443 
## [70] train-rmse:2.148498 
## [71] train-rmse:2.146353 
## [72] train-rmse:2.143242 
## [73] train-rmse:2.141752 
## [74] train-rmse:2.137971 
## [75] train-rmse:2.136961 
## [76] train-rmse:2.134868 
## [77] train-rmse:2.133328 
## [78] train-rmse:2.131047 
## [79] train-rmse:2.128061 
## [80] train-rmse:2.127227 
## [81] train-rmse:2.126399 
## [82] train-rmse:2.124638 
## [83] train-rmse:2.122096 
## [84] train-rmse:2.119564 
## [85] train-rmse:2.118588 
## [86] train-rmse:2.117929 
## [87] train-rmse:2.115806 
## [88] train-rmse:2.112749 
## [89] train-rmse:2.112061 
## [90] train-rmse:2.107018 
## [91] train-rmse:2.105560 
## [92] train-rmse:2.103963 
## [93] train-rmse:2.101254 
## [94] train-rmse:2.099965 
## [95] train-rmse:2.097760 
## [96] train-rmse:2.097208 
## [97] train-rmse:2.094832 
## [98] train-rmse:2.093045 
## [99] train-rmse:2.092483 
## [100]    train-rmse:2.090536
\end{verbatim}

\begin{Shaded}
\begin{Highlighting}[]
\CommentTok{\# Make predictions on the training set (you can use a separate test set for predictions)}
\NormalTok{predictions }\OtherTok{\textless{}{-}} \FunctionTok{predict}\NormalTok{(xgb\_model, }\FunctionTok{as.matrix}\NormalTok{(x\_test))}
\NormalTok{rmse\_10 }\OtherTok{\textless{}{-}} \FunctionTok{RMSE}\NormalTok{(predictions, y\_test)}

\NormalTok{rmse\_summary }\OtherTok{\textless{}{-}} \FunctionTok{bind\_rows}\NormalTok{(rmse\_summary,}
                          \FunctionTok{tibble}\NormalTok{(}\AttributeTok{Model =} \StringTok{"Gradient Boosted Model {-} Standard Params"}\NormalTok{,}
                                 \AttributeTok{RMSE =} \FunctionTok{round}\NormalTok{(rmse\_10, }\DecValTok{3}\NormalTok{)))}
\NormalTok{rmse\_summary}
\end{Highlighting}
\end{Shaded}

\begin{verbatim}
## # A tibble: 10 x 2
##    Model                                                     RMSE
##    <chr>                                                    <dbl>
##  1 Linear Reg - Shell Weight                                 2.4 
##  2 Linear Reg - All Predictor Variables                      2.12
##  3 Linear Reg - All Predictor Variables + Interaction Terms  2.06
##  4 Linear Reg w Ridge (L2) Regression                        2.17
##  5 Linear Reg w Lasso (L1) Regression                        2.06
##  6 Elastic Net Regression                                    2.06
##  7 Elastic Net Regression - Best Alpha                       2.06
##  8 Random Forest - Default Params                            2.04
##  9 Random Forest - Tuned Params                              2.03
## 10 Gradient Boosted Model - Standard Params                  2.08
\end{verbatim}

This isn't looking quite as good as the random forest model, but let's
try tuning parameters anyway:

\begin{Shaded}
\begin{Highlighting}[]
\CommentTok{\# Define hyperparameter grid}
\NormalTok{param\_grid }\OtherTok{\textless{}{-}} \FunctionTok{expand.grid}\NormalTok{(}
  \AttributeTok{eta =} \FunctionTok{c}\NormalTok{(}\FloatTok{0.01}\NormalTok{, }\FloatTok{0.1}\NormalTok{, }\FloatTok{0.3}\NormalTok{),}
  \AttributeTok{max\_depth =} \FunctionTok{c}\NormalTok{(}\DecValTok{3}\NormalTok{, }\DecValTok{6}\NormalTok{, }\DecValTok{9}\NormalTok{),}
  \AttributeTok{gamma =} \FunctionTok{c}\NormalTok{(}\DecValTok{0}\NormalTok{, }\FloatTok{0.1}\NormalTok{, }\FloatTok{0.2}\NormalTok{),}
  \AttributeTok{subsample =} \FunctionTok{c}\NormalTok{(}\FloatTok{0.8}\NormalTok{, }\FloatTok{1.0}\NormalTok{),}
  \AttributeTok{colsample\_bytree =} \FunctionTok{c}\NormalTok{(}\FloatTok{0.8}\NormalTok{, }\FloatTok{1.0}\NormalTok{),}
  \AttributeTok{min\_child\_weight =} \FunctionTok{c}\NormalTok{(}\DecValTok{1}\NormalTok{, }\DecValTok{3}\NormalTok{, }\DecValTok{5}\NormalTok{)}
\NormalTok{)}

\CommentTok{\# Function to train and evaluate model for a given set of hyperparameters}

\NormalTok{train\_and\_evaluate }\OtherTok{\textless{}{-}} \ControlFlowTok{function}\NormalTok{(params, train\_data, test\_data) \{}
  \CommentTok{\# Convert non{-}numeric columns to numeric using one{-}hot encoding}
\NormalTok{  train\_data\_matrix }\OtherTok{\textless{}{-}} \FunctionTok{model.matrix}\NormalTok{(}\SpecialCharTok{\textasciitilde{}}\NormalTok{ . }\SpecialCharTok{{-}} \DecValTok{1}\NormalTok{, }\AttributeTok{data =}\NormalTok{ train\_data)}
\NormalTok{  test\_data\_matrix }\OtherTok{\textless{}{-}} \FunctionTok{model.matrix}\NormalTok{(}\SpecialCharTok{\textasciitilde{}}\NormalTok{ . }\SpecialCharTok{{-}} \DecValTok{1}\NormalTok{, }\AttributeTok{data =}\NormalTok{ test\_data)}
  
  \CommentTok{\# Create DMatrix for training and testing data}
\NormalTok{  dtrain }\OtherTok{\textless{}{-}} \FunctionTok{xgb.DMatrix}\NormalTok{(}\AttributeTok{data =}\NormalTok{ train\_data\_matrix, }\AttributeTok{label =}\NormalTok{ train\_data}\SpecialCharTok{$}\NormalTok{Age)}
\NormalTok{  dtest }\OtherTok{\textless{}{-}} \FunctionTok{xgb.DMatrix}\NormalTok{(}\AttributeTok{data =}\NormalTok{ test\_data\_matrix, }\AttributeTok{label =}\NormalTok{ test\_data}\SpecialCharTok{$}\NormalTok{Age, }\AttributeTok{missing =} \ConstantTok{NA}\NormalTok{, }\AttributeTok{nthread =} \DecValTok{1}\NormalTok{)}
  
  \CommentTok{\# Train the model}
\NormalTok{  xgb\_model }\OtherTok{\textless{}{-}} \FunctionTok{xgboost}\NormalTok{(}\AttributeTok{params =}\NormalTok{ params, }\AttributeTok{data =}\NormalTok{ dtrain, }\AttributeTok{nrounds =} \DecValTok{100}\NormalTok{, }\AttributeTok{verbose =} \DecValTok{0}\NormalTok{)}
  
  \CommentTok{\# Make predictions on the test set}
\NormalTok{  predictions }\OtherTok{\textless{}{-}} \FunctionTok{predict}\NormalTok{(xgb\_model, }\AttributeTok{newdata =}\NormalTok{ dtest)}
  
  \CommentTok{\# Evaluate the model}
\NormalTok{  rmse }\OtherTok{\textless{}{-}} \FunctionTok{sqrt}\NormalTok{(}\FunctionTok{mean}\NormalTok{((predictions }\SpecialCharTok{{-}}\NormalTok{ test\_data}\SpecialCharTok{$}\NormalTok{Age)}\SpecialCharTok{\^{}}\DecValTok{2}\NormalTok{))}
  
  \FunctionTok{return}\NormalTok{(}\FunctionTok{data.frame}\NormalTok{(params, rmse))}
\NormalTok{\}}

\CommentTok{\# Apply the function to all combinations of hyperparameters}
\CommentTok{\# }\AlertTok{WARNING}\CommentTok{: This portion of the code may take 1{-}5 mins to run }
\CommentTok{\# depending on your machine capabilities}
\NormalTok{results }\OtherTok{\textless{}{-}} \FunctionTok{apply}\NormalTok{(param\_grid, }\DecValTok{1}\NormalTok{, }\ControlFlowTok{function}\NormalTok{(row) \{}
\NormalTok{  params }\OtherTok{\textless{}{-}} \FunctionTok{as.list}\NormalTok{(row)}
\NormalTok{  train\_data }\OtherTok{\textless{}{-}}\NormalTok{ train\_set\_continuous}
\NormalTok{  test\_data }\OtherTok{\textless{}{-}}\NormalTok{ test\_set\_continuous}
  \FunctionTok{train\_and\_evaluate}\NormalTok{(params, train\_data, test\_data)}
\NormalTok{\})}


\CommentTok{\# Combine results into a data frame}
\NormalTok{results\_df }\OtherTok{\textless{}{-}} \FunctionTok{do.call}\NormalTok{(rbind, results)}
\end{Highlighting}
\end{Shaded}

Now, let's use the best parameter selection and run another model:

\begin{Shaded}
\begin{Highlighting}[]
\CommentTok{\# Find the best hyperparameters}
\NormalTok{best\_params }\OtherTok{\textless{}{-}}\NormalTok{ results\_df[}\FunctionTok{which.min}\NormalTok{(results\_df}\SpecialCharTok{$}\NormalTok{rmse), ]}
\FunctionTok{print}\NormalTok{(best\_params)}
\end{Highlighting}
\end{Shaded}

\begin{verbatim}
##     eta max_depth gamma subsample colsample_bytree min_child_weight       rmse
## 275 0.1         6     0       0.8                1                5 0.02739996
\end{verbatim}

\begin{Shaded}
\begin{Highlighting}[]
\CommentTok{\# Let\textquotesingle{}s try running our model again with these "best" parameters}
\NormalTok{params }\OtherTok{\textless{}{-}} \FunctionTok{list}\NormalTok{(}
  \AttributeTok{objective =} \StringTok{"reg:squarederror"}\NormalTok{,  }\CommentTok{\# Regression task}
  \AttributeTok{eval\_metric =} \StringTok{"rmse"}\NormalTok{,            }\CommentTok{\# Evaluation metric (Root Mean Squared Error)}
  \AttributeTok{max\_depth =} \DecValTok{3}\NormalTok{,                 }\CommentTok{\# Maximum depth of a tree}
  \AttributeTok{eta =} \FloatTok{0.1}\NormalTok{,     }
  \AttributeTok{gamma =} \DecValTok{0}\NormalTok{,}
  \AttributeTok{subsample =} \FloatTok{0.8}\NormalTok{,}
  \AttributeTok{colsample\_bytree =} \DecValTok{1}\NormalTok{,}
  \AttributeTok{min\_child\_weight =} \DecValTok{1}
\NormalTok{)}

\NormalTok{xgb\_model }\OtherTok{\textless{}{-}} \FunctionTok{xgboost}\NormalTok{(}\AttributeTok{data =}\NormalTok{ dtrain, }\AttributeTok{params =}\NormalTok{ params, }\AttributeTok{nrounds =} \DecValTok{100}\NormalTok{)}
\end{Highlighting}
\end{Shaded}

\begin{verbatim}
## [1]  train-rmse:10.344024 
## [2]  train-rmse:9.372717 
## [3]  train-rmse:8.504990 
## [4]  train-rmse:7.725426 
## [5]  train-rmse:7.030474 
## [6]  train-rmse:6.413966 
## [7]  train-rmse:5.860822 
## [8]  train-rmse:5.371029 
## [9]  train-rmse:4.936015 
## [10] train-rmse:4.553099 
## [11] train-rmse:4.217186 
## [12] train-rmse:3.920132 
## [13] train-rmse:3.664126 
## [14] train-rmse:3.443510 
## [15] train-rmse:3.252543 
## [16] train-rmse:3.083449 
## [17] train-rmse:2.937525 
## [18] train-rmse:2.812477 
## [19] train-rmse:2.704692 
## [20] train-rmse:2.613389 
## [21] train-rmse:2.536005 
## [22] train-rmse:2.469200 
## [23] train-rmse:2.413274 
## [24] train-rmse:2.368845 
## [25] train-rmse:2.329492 
## [26] train-rmse:2.292264 
## [27] train-rmse:2.262583 
## [28] train-rmse:2.237499 
## [29] train-rmse:2.214080 
## [30] train-rmse:2.196983 
## [31] train-rmse:2.181083 
## [32] train-rmse:2.165218 
## [33] train-rmse:2.150994 
## [34] train-rmse:2.138645 
## [35] train-rmse:2.128008 
## [36] train-rmse:2.120572 
## [37] train-rmse:2.112462 
## [38] train-rmse:2.104686 
## [39] train-rmse:2.098263 
## [40] train-rmse:2.093470 
## [41] train-rmse:2.091004 
## [42] train-rmse:2.085045 
## [43] train-rmse:2.081516 
## [44] train-rmse:2.076096 
## [45] train-rmse:2.073425 
## [46] train-rmse:2.068119 
## [47] train-rmse:2.063194 
## [48] train-rmse:2.058832 
## [49] train-rmse:2.054371 
## [50] train-rmse:2.052253 
## [51] train-rmse:2.048445 
## [52] train-rmse:2.047259 
## [53] train-rmse:2.043429 
## [54] train-rmse:2.040015 
## [55] train-rmse:2.033654 
## [56] train-rmse:2.031453 
## [57] train-rmse:2.027265 
## [58] train-rmse:2.023728 
## [59] train-rmse:2.019572 
## [60] train-rmse:2.015710 
## [61] train-rmse:2.011170 
## [62] train-rmse:2.007392 
## [63] train-rmse:2.003972 
## [64] train-rmse:1.999388 
## [65] train-rmse:1.997475 
## [66] train-rmse:1.993854 
## [67] train-rmse:1.988111 
## [68] train-rmse:1.984695 
## [69] train-rmse:1.982568 
## [70] train-rmse:1.979570 
## [71] train-rmse:1.978852 
## [72] train-rmse:1.975535 
## [73] train-rmse:1.971920 
## [74] train-rmse:1.970679 
## [75] train-rmse:1.969916 
## [76] train-rmse:1.968141 
## [77] train-rmse:1.966703 
## [78] train-rmse:1.966153 
## [79] train-rmse:1.963569 
## [80] train-rmse:1.960514 
## [81] train-rmse:1.957587 
## [82] train-rmse:1.956661 
## [83] train-rmse:1.955130 
## [84] train-rmse:1.953019 
## [85] train-rmse:1.950258 
## [86] train-rmse:1.947761 
## [87] train-rmse:1.944704 
## [88] train-rmse:1.941611 
## [89] train-rmse:1.938361 
## [90] train-rmse:1.934661 
## [91] train-rmse:1.932251 
## [92] train-rmse:1.930488 
## [93] train-rmse:1.928848 
## [94] train-rmse:1.926655 
## [95] train-rmse:1.923426 
## [96] train-rmse:1.920168 
## [97] train-rmse:1.917385 
## [98] train-rmse:1.914795 
## [99] train-rmse:1.912192 
## [100]    train-rmse:1.909495
\end{verbatim}

\begin{Shaded}
\begin{Highlighting}[]
\NormalTok{predictions }\OtherTok{\textless{}{-}} \FunctionTok{predict}\NormalTok{(xgb\_model, }\FunctionTok{as.matrix}\NormalTok{(x\_test))}
\NormalTok{rmse\_11 }\OtherTok{\textless{}{-}} \FunctionTok{RMSE}\NormalTok{(predictions, y\_test)}
\NormalTok{rmse\_11}
\end{Highlighting}
\end{Shaded}

\begin{verbatim}
## [1] 2.070017
\end{verbatim}

\begin{Shaded}
\begin{Highlighting}[]
\NormalTok{rmse\_summary }\OtherTok{\textless{}{-}} \FunctionTok{bind\_rows}\NormalTok{(rmse\_summary,}
                          \FunctionTok{tibble}\NormalTok{(}\AttributeTok{Model =} \StringTok{"Gradient Boosted Model {-} Tuned Params"}\NormalTok{,}
                                 \AttributeTok{RMSE =}\NormalTok{ rmse\_11))}
\NormalTok{rmse\_summary}
\end{Highlighting}
\end{Shaded}

\begin{verbatim}
## # A tibble: 11 x 2
##    Model                                                     RMSE
##    <chr>                                                    <dbl>
##  1 Linear Reg - Shell Weight                                 2.4 
##  2 Linear Reg - All Predictor Variables                      2.12
##  3 Linear Reg - All Predictor Variables + Interaction Terms  2.06
##  4 Linear Reg w Ridge (L2) Regression                        2.17
##  5 Linear Reg w Lasso (L1) Regression                        2.06
##  6 Elastic Net Regression                                    2.06
##  7 Elastic Net Regression - Best Alpha                       2.06
##  8 Random Forest - Default Params                            2.04
##  9 Random Forest - Tuned Params                              2.03
## 10 Gradient Boosted Model - Standard Params                  2.08
## 11 Gradient Boosted Model - Tuned Params                     2.07
\end{verbatim}

The tuned boosted model did improve RMSE slightly over the initial
boosted model; however, it may have been too granular since the random
forest models performed better overall.

\hypertarget{summary}{%
\subsection{Summary}\label{summary}}

Within the scope of this project, our tuned random forest model is the
winner!

Let's take a look at a sorted version of our RMSE Summary table:

\begin{Shaded}
\begin{Highlighting}[]
\NormalTok{rmse\_summary\_sorted }\OtherTok{\textless{}{-}}\NormalTok{ rmse\_summary }\SpecialCharTok{\%\textgreater{}\%}
  \FunctionTok{arrange}\NormalTok{(RMSE)}

\NormalTok{rmse\_summary\_sorted}
\end{Highlighting}
\end{Shaded}

\begin{verbatim}
## # A tibble: 11 x 2
##    Model                                                     RMSE
##    <chr>                                                    <dbl>
##  1 Random Forest - Tuned Params                              2.03
##  2 Random Forest - Default Params                            2.04
##  3 Linear Reg w Lasso (L1) Regression                        2.06
##  4 Elastic Net Regression                                    2.06
##  5 Elastic Net Regression - Best Alpha                       2.06
##  6 Linear Reg - All Predictor Variables + Interaction Terms  2.06
##  7 Gradient Boosted Model - Tuned Params                     2.07
##  8 Gradient Boosted Model - Standard Params                  2.08
##  9 Linear Reg - All Predictor Variables                      2.12
## 10 Linear Reg w Ridge (L2) Regression                        2.17
## 11 Linear Reg - Shell Weight                                 2.4
\end{verbatim}

We can likely assume that the random forest model was the best fit for
our data due to plenty of factors; however, overall, the abalone dataset
is a relatively simple one with only nine features (including the target
variable), and may have been generalized better with a random forest
model or linear regression (with all variables and adjusted inputs) over
a more complex boosted model. In addition, most of the data in the
abalone dataset seemed to have linear relationships, giving random
forest and linear regression the edge over a gradient boosted model.
Obviously, the simple linear regression model was too simple (likely
underfit the data). Since we had issues with multicollinearity in the
dataset and therefore, some of the variables were redundant and not
contributing much to the predictive power, elastic net's ability to
perform variable selection may have been advantageous over ridge
regression's approach to simply shrinking all variable coefficients.

\hypertarget{reflections-and-opportunities-for-improvement}{%
\subsection{Reflections and Opportunities for
Improvement}\label{reflections-and-opportunities-for-improvement}}

I went about this problem assuming Age was a continuous variable.
However, since the age range of abalones in the dataset is 2.5 - 30.5,
incremented by 0.5 for each age group, this could have easily been a
classification problem and machine learning models could have been
tailored to a classified result. For example, using cluster methods such
as kNN would have been a good application to assigning an age
``category'' to an abalone using the other eight features in the dataset
as predictors. If I had used this approach, I would have been able to
run confusion matrices, which may have also provided better metrics of
how my model was performing than just using RMSE as an indicator.

I was tentative to remove more observations from the dataset after
cleaning since it already only had around 4,000 observations and 9
features, but another opportunity for improvement could have been better
identification and adjustment of outliers in the dataset, especially
since linear relationships were observed throughout. Removing outliers
could have helped to reduce issues associated with noise (such as
overfitting) and improve predictive power of my models.

Overall, the abalone data sample size was likely the the biggest
bottleneck to achieving better machine learning models and results. More
than anything, I think my algorithms and any future algorithms run on
this dataset would benefit highly from more observations to predict
from. It would be interesting to look into whether there is a similar
quality dataset (more recent than 1995) with greater observations and
features of abalone traits.

\end{document}
